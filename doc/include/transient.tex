The \definition{transient temperature analysis} (TTA) delivers a set of curves, one for each of the thermal nodes, that describe the thermal behaviour of the system through the time horizon of a dynamic power profile $\pprofile{\dyn}$. Since the given power profile is a mean (nominal) profile, we denote it by $\mpprofile{\dyn}$.

\subsection{Problem Formulation}
Given:
\begin{itemize}
  \item A multiprocessor system $\system$.
  \item A mean dynamic power profile $\mpprofile{\dyn}$.
  \item The variation ratio matrix $\mratio_\dyn$ and correlation matrix $\corr{\vpower_\dyn}$ of the dynamic power.
  \item A mean leakage power $\vmean{\leak}$.
  \item The covariance matrix $\cov{\vpower_\leak}$ of the leakage power.
  \item The covariance matrix $\cov{\noise}$ of the environmental noise.
\end{itemize}

Find:
\begin{itemize}
  \item The probability distribution of the temperature profile $\tprofile{}$ with respect to $\mpprofile{\dyn}$.
\end{itemize}

First, the sources of uncertainties are modeled separately (\secref{tta-process-variation} and \secref{tta-noise}), and then they are joined together into a single model (\secref{tta-process-variation-noise}).

\subsection{Solution with Process Variation} \seclabel{tta-process-variation}
Additional assumptions:
\begin{itemize}
  \item $\power_{ij}$ are distributed normally with known variance $\deviation{ij}^2 \; \forall i \in \nodeindex, j \in \timeindex$.
  \item $\power_{i_1 j_1}$ and $\power_{i_2 j_2}$ are independent $\forall i_1, i_2 \in \nodeindex,  i_1 \neq i_2$ and $\forall j_1, j_2 \in \timeindex$.
\end{itemize}

The nominal power dissipation is assumed to be fixed within each time interval $\timeinterval_i$, resulting in the constant term $\vpower$ instead of $\vpower(t)$ in \equref{fourier}:
\begin{equation} \equlabel{fourier-constant-power}
  \mcapacitance \frac{d\vtemperature(\time)}{d\time} + \mconductance \vtemperature(\time) = \vpower
\end{equation}
The solution of \equref{fourier-constant-power} with constant power is the following:
\[
  \vtemperature(\time) = \m{A}(\time) \vtemperature_0 + \m{B}(\time) \vpower
\]
where:
\begin{align}
  & \m{A}(\time) = \emcg{\time} \equlabel{a} \\
  & \m{B}(\time) = -(\cg)^{-1}(\emcg{\time} - \mone) \mcapacitance^{-1} \equlabel{b}
\end{align}
The following recurrence is used to sequentially calculate temperature for all time intervals of the power profile:
\begin{align}
  \vtemperature_{i + 1} & = \m{A}_i \vtemperature_i + \m{B}_i \vpower_i \nonumber \\
  & = \m{A}_i \vtemperature_i + \m{B}_i (\vmean{i} + \ddeviation{i} \vstdnormal), \sep i \in \timeindex \equlabel{tta-fourier-recurrence}
\end{align}
where we denote $\vtemperature_i = \vtemperature_{\time_i}$, $\vpower_i = \vpower_{\time_i}$, $\m{A}_i = \m{A}(\timeinterval_i)$, and $\m{B}_i = \m{B}(\timeinterval_i)$. Here, similar to \secref{sss-solution}, the power dissipation for all time intervals is modeled using a single \msnrv\ $\vstdnormal \sim \normal(\vzero, \mone)$ (see \equref{power-rv}) that takes into consideration the uniform deviation from the nominal value, discussed in \secref{process-variation}. In this case, $\vtemperature_i$ and $\vpower_i$ are no longer independent, since they are linear transformation of the same \mrv. It can be shown that a linear combination $\m{A} \v{X} + \m{B} \v{Y}$ of two dependent \mnrvs\ $\v{X} \sim \normal(\expectation{\v{X}}, \covariance{\v{X}})$ and $\v{Y} \sim \normal(\expectation{\v{X}}, \covariance{\v{X}})$, which is what we have in \equref{tta-fourier-recurrence}, is a \mnrv\ $\v{Z} \sim \normal(\m{A} \expectation{\v{X}} + \m{B} \expectation{\v{X}}, \m{A} \covariance{\v{X}} \m{A}^T + \m{B} \covariance{\v{Y}} \m{B}^T + \m{A} \covariance{\v{X}, \v{Y}} \m{B}^T + \m{B} \covariance{\v{X}, \v{Y}}^T \m{A}^T)$, where $\covariance{\v{X}, \v{Y}}$ is the cross-covariance matrix of the variables $\v{X}$ and $\v{Y}$. Consequently:
\begin{align*}
  & \vtemperature_{i + 1} \sim \normal(\expectation{\vtemperature_{i + 1}}, \covariance{\vtemperature_{i + 1}}), \sep i \in \timeindex \\
  & \expectation{\vtemperature_{i + 1}} = \m{A}_i \expectation{\vtemperature_i} + \m{B}_i \vmean{i} \\
  & \covariance{\vtemperature_{i + 1}} = \m{A}_i \covariance{\vtemperature_i} \m{A}_i^T + \m{B}_i \ddeviation{i}^2 \m{B}_i^T \\
  & {} \qquad \qquad + \m{A}_i \covariance{\vtemperature_i, \vpower_i} \m{B}_i^T + \m{B}_i \covariance{\vtemperature_i, \vpower_i}^T \m{A}_i^T
\end{align*}
It can be shown that the cross-covariance matrix is computed according to the following expression:
\begin{align}
  & \covariance{\vtemperature_0, \vpower_0} = \mzero \nonumber \\
  & \covariance{\vtemperature_i, \vpower_i} = \sum_{j = 0}^{i - 1} \am{A}_{(i - 1)(j + 1)} \m{B}_j \ddeviation{j} \ddeviation{i}, \sep i \in \timeindex \setminus \{0\}  \equlabel{cross-covariance}
\end{align}
where the following shorthand for the matrix product is introduced:
\[
  \am{A}_{ij} = \begin{cases}
    \mone & \mbox{if } j > i \\
    \m{A}_i \m{A}_{i - 1} \dots \m{A}_{j + 1} \m{A}_j & \mbox{otherwise}
  \end{cases}
\]

Alternatively, non-recurrent expressions can be obtained by performing an iterative repetition of \equref{tta-fourier-recurrence}:
\begin{equation} \equlabel{tta-fourier-process-variation-non-recurrent}
  \vtemperature_{i + 1} = \am{A}_{i0} \vtemperature_0 + \v{F}_i + \v{H}_i \vstdnormal, \sep i \in \timeindex
\end{equation}
where:
\begin{align}
  & \v{F}_i = \sum_{j = 0}^{i} \am{A}_{i(j+1)} \m{B}_j \expectation{\vpower_j} = \sum_{j = 0}^{i} \am{A}_{i(j+1)} \m{B}_j \vmean{j} \equlabel{f} \\
  & \v{H}_i = \sum_{j = 0}^{i} \am{A}_{i(j+1)} \m{B}_j \ddeviation{j} \equlabel{h}
\end{align}
Therefore:
\begin{align*}
  & \vtemperature_{i + 1} \sim \normal(\expectation{\vtemperature_{i + 1}}, \covariance{\vtemperature_{i + 1}}), \sep i \in \timeindex \\
  & \expectation{\vtemperature_{i + 1}} = \am{A}_{i0} \expectation{\vtemperature_0} + \v{F}_i \\
  & \covariance{\vtemperature_{i + 1}} = \am{A}_{i0} \covariance{\vtemperature_0} \am{A}_{i0}^T + \v{H}_i \v{H}_i^T
\end{align*}
If the initial temperature vector is assumed to be deterministic, $\expectation{\vtemperature_0} \equiv \vtemperature_0$ and $\covariance{\vtemperature_0} = \mzero$. Also, it is worth mentioning that, due to the fact that we let $\vtemperature(t) \equiv \vtemperature(t) - \vtemperature_\amb$, $\vtemperature_0$ is always zero, however, we prefer to keep it for the equations to be consistent.


\subsection{Solution with Noise} \seclabel{tta-noise}
Suppose the system at each moment of time is affected by a random noise, symbolically denoted by $\noise$. The nominal parameters of the platform and ambience are assumed to be known, however, due to the noise (caused by, for instance, varying environmental conditions), the actual dynamic power dissipation and, consequently, temperature fluctuations are uncertain.

Additional assumptions:
\begin{itemize}
  \item $\power_{ij}$ are deterministic $\forall i \in \nodeindex, j \in \timeindex$.
  \item $\noise$ is the white noise.
  \item The covariance matrix $\covariance{\noise}$ of the noise is known.
\end{itemize}

The presence of the noise is modeled with an additional term, $\noise$, in the right-hand side of \equref{fourier-constant-power}. Due to the assumptions, $\noise$ is a \mnrv\ with $\nodecount$ components (one for each thermal node), i.e., $\noise \sim \normal(\vzero, \covariance{\noise})$. Since the covariance matrix is a real symmetric matrix, it can be diagonalized using the eigenvalue decomposition as $\covariance{\noise} = \m{U} \m{\Lambda} \m{U}^T$, where $\m{U}$ is a orthogonal matrix of the eigenvectors and $\m{\Lambda}$ is a diagonal matrix of the eigenvalues \cite{press2007}. Consequently, the noise is decomposed into $\noise = \mnoisedeviation \vstdnormal_\time$, where $\mnoisedeviation = \m{U} \m{\Lambda}^{1/2}$ and the stochastic process $\{ \vstdnormal_\time \}_{\time \in \timeset}$ is the standard $\nodecount$-dimensional white noise \cite{oksendal2003}, a set of vectors of independent \snrvs, i.e., $\vstdnormal_\time \sim \normal(\vzero, \mone)$\footnote{$\mnoisedeviation \vstdnormal_\time = \m{U} \m{\Lambda}^{1/2} \vstdnormal_t \sim \normal(\vzero, \m{U} \m{\Lambda} \m{U}^T = \covariance{\noise})$.}. Hence, we have the following stochastic process:
\begin{equation} \equlabel{fourier-noise}
  \mcapacitance \frac{d\vtemperature_\time}{d\time} + \mconductance \vtemperature_\time = \vpower + \mnoisedeviation \vstdnormal_\time
\end{equation}
\equref{fourier-noise} can be rewritten in the differential form:
\begin{equation} \equlabel{fourier-wiener}
  d\vtemperature_\time = \mcapacitance^{-1} (\vpower - \mconductance \vtemperature_\time)d\time + \mcapacitance^{-1} \mnoisedeviation d\m{W}_\time
\end{equation}
where $d\m{W}_\time = \vstdnormal_\time d\time$ with $\{ \m{W}_\time \}_{\time \in \timeset}$ being the Wiener process or Brownian motion \cite{oksendal2003}. The integration of \equref{fourier-wiener} cannot be done in the regular Riemann-Stieltjes sense, since the Brownian motion is nowhere differentiable, therefore, a special calculus should be applied. We follow the It\^{o} interpretation \cite{oksendal2003} of the stochastic differential equation in \equref{fourier-wiener}. It can be seen that the stochastic process $\{ \vtemperature_\time \}_{\time \in \timeset}$ is a multidimensional standard (i.e., Gaussian) Ornstein-Uhlenbeck process \cite{kloeden1992} driven by the Brownian motion. In order to solve \equref{fourier-wiener}, we apply the It\^{o} formula \cite{oksendal2003} to the function $\ecg{}{t} \vtemperature_\time$, where $e^\m{A}$ denotes the matrix exponential of $\m{A}$, and obtain:
\begin{align*}
  d(\ecgt \mtemperature_\time) & = \cg \ecgt \vtemperature_\time d\time + \ecgt d\vtemperature_\time \\
  & = \ecgt \mcapacitance^{-1} (\vpower d\time + \mnoisedeviation d\v{W}_t)
\end{align*}
The solution can be found by taking integrals on both sides:
\begin{align} \equlabel{solution-full}
  \vtemperature_\time = & \emcgt{} \vtemperature_0 - \\
    & (\cg)^{-1}(\emcgt{} - \mone) \mcapacitance^{-1} \vpower + \nonumber \\
    & \int_0^\time \ecg{}{(s - \time)} \mcapacitance^{-1} \mnoisedeviation \; d\m{W}_s \nonumber
\end{align}
To simplify the further derivation, we introduce the following notation:
\[
  \v{D}_\time = \int_0^{\time} \ecg{}{(s - \time)} \mcapacitance^{-1} \mnoisedeviation \: d\m{W}_s
\]
Therefore, taking into consideration \equref{a} and \equref{b}, \equref{solution-full} can be rewritten:
\[
  \vtemperature_\time = \m{A}(\time) \vtemperature_0 + \m{B}(\time) \vpower + \v{D}_\time
\]
Since the Wiener process has independent normally distributed increments, an integration with respect to it, i.e., $\int_0^\time f(s) dW_s$, is a \nrv. It can also be shown that in this case the mean is zero and variance is equal to $\int_0^\time f^2(s) ds$. Hence, $\v{D}_\time \sim \normal(\vzero, \covariance{\v{D}_\time})$, i.e., $\v{D}_\time$ is a \mnrv\ with a zero mean vector and the following covariance matrix:
\begin{align}
  \covariance{\v{D}_\time} & = \int_0^\time \ecg{2}{(s - \time)} (\mcapacitance^{-1} \mnoisedeviation)^2 \; ds \nonumber \\
  & = (2 \cg)^{-1} (\mone - \emcgt{2}) (\mcapacitance^{-1} \mnoisedeviation)^2 \equlabel{covariance-d}
\end{align}
Consequently, $\m{B}(\time) \vpower + \v{D}_\time \sim \normal(\m{B}(\time) \vpower, \: \covariance{\v{D}_\time})$.

The solution given by \equref{solution-full} is applicable for one time interval with constant $\vpower$. In order to model the whole duration $\period$ of the power profile, the computations should be performed for each of the subintervals sequentially. Therefore, we have the following recurrent expression:
\begin{equation} \equlabel{tta-fourier-recurrence-noise}
  \vtemperature_{i + 1} = \m{A}_i \vtemperature_i + \m{B}_i \vpower_i + \m{D}_i, \sep i \in \timeindex
\end{equation}
where $\m{D}_i = \m{D}_{\timeinterval_i}$.

Taking into consideration the properties of the Wiener process, $\vtemperature_i$ is independent of $\m{D}_i$. Therefore, $\vtemperature_{i+1}$ is a \mnrv:
\begin{align*}
  & \vtemperature_{i + 1} \sim \normal(\expectation{\vtemperature_{i + 1}}, \covariance{\vtemperature_{i + 1}}), \sep i \in \timeindex \\
  & \expectation{\vtemperature_{i+1}} = \m{A}_i \expectation{\vtemperature_i} + \m{B}_i \vpower_i \\
  & \covariance{\vtemperature_{i+1}} = \m{A}_i \covariance{\vtemperature_i} \m{A}^T_i + \covariance{\m{D}_i}
\end{align*}

Alternatively, non-recurrent expressions can be obtained by performing an iterative repetition of \equref{tta-fourier-recurrence-noise}:
\begin{equation} \equlabel{tta-fourier-noise-non-recurrent}
  \vtemperature_{i + 1} = \am{A}_{i0} \vtemperature_0 + \m{F}_i + \v{K}_i, \sep i \in \timeindex
\end{equation}
where $\m{F}_i$ is computed according to \equref{f} with $\expectation{\vpower_i} \equiv \vpower_i$ and:
\begin{equation} \equlabel{k}
  \v{K}_i = \sum_{j = 0}^{i} \am{A}_{i(j+1)} \m{D}_j
\end{equation}
Therefore:
\begin{align*}
  & \vtemperature_{i + 1} \sim \normal(\expectation{\vtemperature_{i + 1}}, \covariance{\vtemperature_{i + 1}}), \sep i \in \timeindex \\
  & \expectation{\vtemperature_{i + 1}} = \am{A}_{i0} \expectation{\vtemperature_0} + \m{F}_i \\
  & \covariance{\vtemperature_{i + 1}} = \am{A}_{i0} \covariance{\vtemperature_0} \am{A}_{i0}^T + \m{M}_i
\end{align*}
where for convenience we denote:
\begin{equation} \equlabel{m}
  \m{M}_i = \sum_{j = 0}^{i} \am{A}_{i(j+1)} \covariance{\m{D}_j} \am{A}_{i(j+1)}^T
\end{equation}


\subsection{Solution with Process Variation and Noise} \seclabel{tta-process-variation-noise}
In this section we combine models presented in \secref{tta-noise} and \secref{tta-process-variation}. The starting point is the following recurrence:
\begin{align}
  \vtemperature_{i + 1} & = \m{A}_i \vtemperature_i + \m{B}_i \vpower_i + \m{D}_i \nonumber \\
  & = \m{A}_i \vtemperature_i + \m{B}_i(\vmean{i} + \ddeviation{i} \vstdnormal) + \m{D}_i, \sep i \in \timeindex \equlabel{fourier-recurrence-noise-process-variation}
\end{align}
Here $\m{D}_i \; \forall i \in \timeindex$ and $\vstdnormal$ are independent \mnrvs. Thus:
\begin{align*}
  & \vtemperature_{i + 1} \sim \normal(\expectation{\vtemperature_{i + 1}}, \covariance{\vtemperature_{i + 1}}), \sep i \in \timeindex \\
  & \expectation{\vtemperature_{i + 1}} = \m{A}_i \expectation{\vtemperature_i} + \m{B}_i \vmean{i} \\
  & \covariance{\vtemperature_{i + 1}} = \m{A}_i \covariance{\vtemperature_i} \m{A}_i^T + \m{B}_i \ddeviation{i}^2 \m{B}_i^T \\
  & {} \qquad + \m{A}_i \covariance{\vtemperature_i, \vpower_i} \m{B}_i^T + \m{B}_i \covariance{\vtemperature_i, \vpower_i}^T \m{A}_i^T + \covariance{\m{D}_i}
\end{align*}
where $\covariance{\vtemperature_i, \vpower_i}$ is computed using \equref{cross-covariance}.

Alternatively, non-recurrent expressions can be obtained by performing an iterative repetition of \equref{fourier-recurrence-noise-process-variation}:
\begin{equation} \equlabel{fourier-non-recurrent-combined}
  \vtemperature_{i + 1} = \am{A}_{i0} \expectation{\vtemperature_0} + \v{F}_i + \v{H}_i \vstdnormal + \v{K}_i, \sep i \in \timeindex
\end{equation}
Therefore:
\begin{align*}
  & \vtemperature_{i + 1} \sim \normal(\expectation{\vtemperature_{i + 1}}, \covariance{\vtemperature_{i + 1}}), \sep i \in \timeindex \\
  & \expectation{\vtemperature_{i + 1}} = \am{A}_i \expectation{\vtemperature_0} + \v{F}_i \\
  & \covariance{\vtemperature_{i + 1}} = \am{A}_{i0} \covariance{\vtemperature_0} \am{A}_{i0}^T + \v{H}_i \v{H}_i^T + \m{M}_i
\end{align*}

