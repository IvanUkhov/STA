The \definition{transient temperature analysis} (TTA) delivers a set of curves, one for each of the thermal nodes, that describe the thermal behaviour of the system through the time horizon of a dynamic power profile $\pprofile{\dyn}$. Since the given power profile is a mean (nominal) profile, we denote it by $\mpprofile{\dyn}$.

\subsection{Problem Formulation}
Given:
\begin{itemize}
  \item A multiprocessor system $\system$.
  \item A mean dynamic power profile $\mpprofile{\dyn}$.
  \item The variation ratio matrix $\mratio_\dyn$ and correlation matrix $\corr{\vpower_\dyn}$ of the dynamic power.
  \item A mean leakage power $\vmean{\leak}$.
  \item The covariance matrix $\cov{\vpower_\leak}$ of the leakage power.
  \item The covariance matrix $\cov{\noise}$ of the environmental noise.
\end{itemize}

Find:
\begin{itemize}
  \item The probability distribution of the temperature profile $\tprofile{}$ with respect to $\mpprofile{\dyn}$.
\end{itemize}

First, the sources of uncertainties are modeled separately (\secref{tta-process-variation} and \secref{tta-noise}), and then they are joined together into a single model (\secref{tta-process-variation-noise}).

\subsection{Solution with Process Variation} \seclabel{tta-process-variation}
Additional assumptions:
\begin{itemize}
  \item $\power_{ij}$ are distributed normally with known variance $\deviation{ij}^2 \; \forall i \in \nodeindex, j \in \timeindex$.
  \item $\power_{i_1 j_1}$ and $\power_{i_2 j_2}$ are independent $\forall i_1, i_2 \in \nodeindex,  i_1 \neq i_2$ and $\forall j_1, j_2 \in \timeindex$.
\end{itemize}

The nominal power dissipation is assumed to be fixed within each time interval $\timeinterval_i$, resulting in the constant term $\vpower$ instead of $\vpower(t)$ in \equref{fourier}:
\begin{equation} \equlabel{fourier-constant-power}
  \mcapacitance \frac{d\vtemperature(\time)}{d\time} + \mconductance \vtemperature(\time) = \vpower
\end{equation}
The solution of \equref{fourier-constant-power} with constant power is the following:
\[
  \vtemperature(\time) = \m{A}(\time) \vtemperature_0 + \m{B}(\time) \vpower
\]
where:
\begin{align}
  & \m{A}(\time) = \emcg{\time} \equlabel{a} \\
  & \m{B}(\time) = -(\cg)^{-1}(\emcg{\time} - \mone) \mcapacitance^{-1} \equlabel{b}
\end{align}
The following recurrence is used to sequentially calculate temperature for all time intervals of the power profile:
\begin{align}
  \vtemperature_{i + 1} & = \m{A}_i \vtemperature_i + \m{B}_i \vpower_i \nonumber \\
  & = \m{A}_i \vtemperature_i + \m{B}_i (\vmean{i} + \ddeviation{i} \vstdnormal), \sep i \in \timeindex \equlabel{tta-fourier-recurrence}
\end{align}
where we denote $\vtemperature_i = \vtemperature_{\time_i}$, $\vpower_i = \vpower_{\time_i}$, $\m{A}_i = \m{A}(\timeinterval_i)$, and $\m{B}_i = \m{B}(\timeinterval_i)$. Here, similar to \secref{sss-solution}, the power dissipation for all time intervals is modeled using a single \msnrv\ $\vstdnormal \sim \normal(\vzero, \mone)$ (see \equref{dynamic-power}) that takes into consideration the uniform deviation from the nominal value, discussed in \secref{uncertainties}. In this case, $\vtemperature_i$ and $\vpower_i$ are no longer independent, since they are linear transformation of the same \mrv. It can be shown that a linear combination $\m{A} \v{X} + \m{B} \v{Y}$ of two dependent \mnrvs\ $\v{X} \sim \normal(\expectation{\v{X}}, \covariance{\v{X}})$ and $\v{Y} \sim \normal(\expectation{\v{X}}, \covariance{\v{X}})$, which is what we have in \equref{tta-fourier-recurrence}, is a \mnrv\ $\v{Z} \sim \normal(\m{A} \expectation{\v{X}} + \m{B} \expectation{\v{X}}, \m{A} \covariance{\v{X}} \m{A}^T + \m{B} \covariance{\v{Y}} \m{B}^T + \m{A} \covariance{\v{X}, \v{Y}} \m{B}^T + \m{B} \covariance{\v{X}, \v{Y}}^T \m{A}^T)$, where $\covariance{\v{X}, \v{Y}}$ is the cross-covariance matrix of the variables $\v{X}$ and $\v{Y}$. Consequently:
\begin{align*}
  & \vtemperature_{i + 1} \sim \normal(\expectation{\vtemperature_{i + 1}}, \covariance{\vtemperature_{i + 1}}), \sep i \in \timeindex \\
  & \expectation{\vtemperature_{i + 1}} = \m{A}_i \expectation{\vtemperature_i} + \m{B}_i \vmean{i} \\
  & \covariance{\vtemperature_{i + 1}} = \m{A}_i \covariance{\vtemperature_i} \m{A}_i^T + \m{B}_i \ddeviation{i}^2 \m{B}_i^T \\
  & {} \qquad \qquad + \m{A}_i \covariance{\vtemperature_i, \vpower_i} \m{B}_i^T + \m{B}_i \covariance{\vtemperature_i, \vpower_i}^T \m{A}_i^T
\end{align*}
It can be shown that the cross-covariance matrix is computed according to the following expression:
\begin{align}
  & \covariance{\vtemperature_0, \vpower_0} = \mzero \nonumber \\
  & \covariance{\vtemperature_i, \vpower_i} = \sum_{j = 0}^{i - 1} \am{A}_{(i - 1)(j + 1)} \m{B}_j \ddeviation{j} \ddeviation{i}, \sep i \in \timeindex \setminus \{0\}  \equlabel{cross-covariance}
\end{align}
where the following shorthand for the matrix product is introduced:
\[
  \am{A}_{ij} = \begin{cases}
    \mone & \mbox{if } j > i \\
    \m{A}_i \m{A}_{i - 1} \dots \m{A}_{j + 1} \m{A}_j & \mbox{otherwise}
  \end{cases}
\]

Alternatively, non-recurrent expressions can be obtained by performing an iterative repetition of \equref{tta-fourier-recurrence}:
\begin{equation} \equlabel{tta-fourier-process-variation-non-recurrent}
  \vtemperature_{i + 1} = \am{A}_{i0} \vtemperature_0 + \v{F}_i + \v{H}_i \vstdnormal, \sep i \in \timeindex
\end{equation}
where:
\begin{align}
  & \v{F}_i = \sum_{j = 0}^{i} \am{A}_{i(j+1)} \m{B}_j \expectation{\vpower_j} = \sum_{j = 0}^{i} \am{A}_{i(j+1)} \m{B}_j \vmean{j} \equlabel{f} \\
  & \v{H}_i = \sum_{j = 0}^{i} \am{A}_{i(j+1)} \m{B}_j \ddeviation{j} \equlabel{h}
\end{align}
Therefore:
\begin{align*}
  & \vtemperature_{i + 1} \sim \normal(\expectation{\vtemperature_{i + 1}}, \covariance{\vtemperature_{i + 1}}), \sep i \in \timeindex \\
  & \expectation{\vtemperature_{i + 1}} = \am{A}_{i0} \expectation{\vtemperature_0} + \v{F}_i \\
  & \covariance{\vtemperature_{i + 1}} = \am{A}_{i0} \covariance{\vtemperature_0} \am{A}_{i0}^T + \v{H}_i \v{H}_i^T
\end{align*}
If the initial temperature vector is assumed to be deterministic, $\expectation{\vtemperature_0} \equiv \vtemperature_0$ and $\covariance{\vtemperature_0} = \mzero$. Also, it is worth mentioning that, due to the fact that we let $\vtemperature(t) \equiv \vtemperature(t) - \vtemperature_\amb$, $\vtemperature_0$ is always zero, however, we prefer to keep it for the equations to be consistent.


\subsection{Solution with Noise} \seclabel{tta-noise}
The solution given by \equref{solution-full} is applicable for one time interval with constant $\vpower$. In order to model the whole duration $\period$ of the power profile, the computations should be performed for each of the subintervals sequentially. Therefore, we have the following recurrent expression:
\begin{equation} \equlabel{tta-fourier-recurrence-noise}
  \vtemperature_{i + 1} = \m{A}_i \vtemperature_i + \m{B}_i \vpower_i + \m{D}_i, \sep i \in \timeindex
\end{equation}
where $\m{D}_i = \m{D}_{\timeinterval_i}$.

Taking into consideration the properties of the Wiener process, $\vtemperature_i$ is independent of $\m{D}_i$. Therefore, $\vtemperature_{i+1}$ is a \mnrv:
\begin{align*}
  & \vtemperature_{i + 1} \sim \normal(\expectation{\vtemperature_{i + 1}}, \covariance{\vtemperature_{i + 1}}), \sep i \in \timeindex \\
  & \expectation{\vtemperature_{i+1}} = \m{A}_i \expectation{\vtemperature_i} + \m{B}_i \vpower_i \\
  & \covariance{\vtemperature_{i+1}} = \m{A}_i \covariance{\vtemperature_i} \m{A}^T_i + \covariance{\m{D}_i}
\end{align*}

Alternatively, non-recurrent expressions can be obtained by performing an iterative repetition of \equref{tta-fourier-recurrence-noise}:
\begin{equation} \equlabel{tta-fourier-noise-non-recurrent}
  \vtemperature_{i + 1} = \am{A}_{i0} \vtemperature_0 + \m{F}_i + \v{K}_i, \sep i \in \timeindex
\end{equation}
where $\m{F}_i$ is computed according to \equref{f} with $\expectation{\vpower_i} \equiv \vpower_i$ and:
\begin{equation} \equlabel{k}
  \v{K}_i = \sum_{j = 0}^{i} \am{A}_{i(j+1)} \m{D}_j
\end{equation}
Therefore:
\begin{align*}
  & \vtemperature_{i + 1} \sim \normal(\expectation{\vtemperature_{i + 1}}, \covariance{\vtemperature_{i + 1}}), \sep i \in \timeindex \\
  & \expectation{\vtemperature_{i + 1}} = \am{A}_{i0} \expectation{\vtemperature_0} + \m{F}_i \\
  & \covariance{\vtemperature_{i + 1}} = \am{A}_{i0} \covariance{\vtemperature_0} \am{A}_{i0}^T + \m{M}_i
\end{align*}
where for convenience we denote:
\begin{equation} \equlabel{m}
  \m{M}_i = \sum_{j = 0}^{i} \am{A}_{i(j+1)} \covariance{\m{D}_j} \am{A}_{i(j+1)}^T
\end{equation}


\subsection{Solution with Process Variation and Noise} \seclabel{tta-process-variation-noise}
In this section we combine models presented in \secref{tta-noise} and \secref{tta-process-variation}. The starting point is the following recurrence for $i \in \timeindex$:
\begin{align} \equlabel{fourier-recurrence-noise-process-variation}
  & \vtemp_{i + 1} = \m{A}_i \vtemp_i + \m{D}_i \nonumber \\
  & {} \qquad + \m{B}_i (\vmean{i} + \ddev{\dyn \; i} \vnormal_\dyn + \ddev{\leak} \vnormal_\leak)
\end{align}
Here $\m{D}_i, \forall i \in \timeindex$, $\vnormal_\dyn$, and $\vnormal_\leak$ are independent \mnrvs. Thus:
\begin{align*}
  & \vtemp_{i + 1} \sim \normal(\expect{\vtemp_{i + 1}}, \cov{\vtemp_{i + 1}}), \sep i \in \timeindex \\
  & \expect{\vtemp_{i + 1}} = \m{A}_i \expect{\vtemp_i} + \m{B}_i \vmean{i} \\
  & \cov{\vtemp_{i + 1}} = \dbl{\m{A}_i, \cov{\vtemp_i}} + \dbl{\m{B}_i, \ddev{\dyn \; i}^2 + \ddev{\leak \; i}^2} + \cov{\m{D}_i} \\
  & {} \qquad + \dbl{\dbl{\m{A}_i, \m{H}_{\dyn \; i} \ddev{\dyn \; i} + \m{H}_{\leak \; i} \ddev{\leak}, \m{B}_i}}
\end{align*}
where $\m{H}_{\dyn \; i}$ and $\m{H}_{\leak \; i}$ are computed using \equref{h-dynamic} and \equref{h-leakage}, respectively.

Alternatively, non-recurrent expressions can be obtained by performing an iterative repetition of \equref{fourier-recurrence-noise-process-variation}:
\begin{align} \equlabel{fourier-non-recurrent-combined}
  & \vtemp_{i + 1} = \am{A}_{i0} \expect{\vtemp_0} + \v{F}_i + \v{L}_i \nonumber \\
  & { } \qquad + \v{H}_{\dyn \; i} \vnormal_\dyn + \v{H}_{\leak \; i} \vnormal_\leak, \sep i \in \timeindex
\end{align}
Therefore:
\begin{align*}
  & \vtemp_{i + 1} \sim \normal(\expect{\vtemp_{i + 1}}, \cov{\vtemp_{i + 1}}), \sep i \in \timeindex \\
  & \expect{\vtemp_{i + 1}} = \am{A}_i \expect{\vtemp_0} + \v{F}_i \\
  & \cov{\vtemp_{i + 1}} = \dbl{\am{A}_{i0}, \cov{\vtemp_0}} + \dbl{\v{H}_{\dyn \; i}, \mone} + \dbl{\v{H}_{\leak \; i}, \mone} + \m{M}_i
\end{align*}

