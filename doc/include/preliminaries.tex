\subsection{Architecture Model}
Consider a multiprocessor system $\system = (\platform, \application, \physics)$. $\platform = \{ \core_i \}_{i \in \coreindex}$ is a platform modeled as a set of programmable processors and buses, which, without loss of generality, we do not treat separately and refer to as cores; $\coreindex = \{0, \dots, \corecount - 1\}$ is an index set. $\application = (\taskset, \edgeset)$ is an application modeled as a direct acyclic task graph, where $\taskset = \{ \task_i \}_{i \in \taskindex}$ is the set of processing and communication tasks with an index set $\taskindex = \{ 0, \dots, \taskcount - 1 \}$, $\edgeset$ is a set of dependencies between them; as with the platform, we do not make any distinction between types of tasks and refer to the whole set as simply tasks. $\physics$ denotes a set of physical characteristics of the platform and ambience, which includes the floorplan of the cores, geometry of the die and thermal package, thermal parameters of the ambience and materials that the die and package are made of.

A mapping of the application $\application$ onto the platform $\platform$ is a function $\mapping: \taskset \to \coreset$ that assigns a core to each task. Let $\executiontime: \taskset \to \nonnegativereal$ and $\starttime: \taskset \to \nonnegativereal$ be functions that determine the execution and start times of tasks, respectively. A tuple $\schedule = (\mapping, \executiontime, \starttime)$ defines a schedule of the application onto the given platform. For simplicity, we denote $\executiontime(\task_i) = \executiontime_{\task_i} = \executiontime_i$ and $\starttime(\task_i) = \starttime_{\task_i} = \starttime_i$.

\subsection{Power and Thermal Models}
The dynamic power dissipation of a task $\task_i \in \taskset$ is denoted by $\power_{\task_i}$. When the application is being executed periodically, this value can vary between actuations, therefore, denote $\power^{(k)}_{\task_i}$ the dynamic power dissipation in the $k$th actuation of the application, where $k \in \natural$.

Given $\physics$, an equivalent thermal RC circuit of the system is constructed \cite{kreith2000}. The circuit is composed of a set of thermal nodes denoted by $\nodeset = \{ \node_i \}_{i \in \nodeindex}$, where $\nodeindex = \{ 0, \dots, \nodecount - 1 \}$ is an index set. Without loss of generality, assume that each core is mapped onto exactly one thermal node; if it is not the case, cores in the floorplan are partitioned into several blocks, each for a single thermal node, and their dynamic power dissipation is split proportionally to the area of the blocks. Let $\circuit: \coreset \injection \nodeset$ be an injective function that performs the mapping between cores and thermal nodes. The nodes that correspond to cores are called active, other nodes are called passive.

The dynamic power dissipation of the constructed circuit is denoted by a vector $\vpower = \vector{\power_i}^T_{i \in \nodeindex}$, where $\power_i$ is the power of the $i$th node. Passive nodes dissipate no power. The corresponding temperature variation is denoted by $\vtemperature = \vector{\temperature_i}^T_{i \in \nodeindex}$, where $\temperature_i$ is the temperature of the $i$th node.

The thermal behaviour of the system is modeled with the following heat equation:
\begin{equation} \equlabel{fourier}
  \mcapacitance \frac{d\vtemperature(\time)}{d\time} + \mconductance (\vtemperature(\time) - \vtemperature_{amb}) = \vpower(\time)
\end{equation}
where $\mcapacitance$ and $\mconductance$ are $\nodecount \times \nodecount$ matrices of the thermal capacitance and conductance, respectively. $\vtemperature_{amb}$ is the ambient temperature. Without loss of generality, let $\vtemperature_{amb} = 0$.

A discrete dynamic power profile of the system during a time interval $\totaltimeinterval$ is defined as a tuple $\powerprofile = (\timepartition, \mpower)$. $\timepartition = \{ \time_i: i = 0, \dots, \stepcount \}$ is a partition of $\totaltimeinterval$ into $\stepcount$ subintervals $\timeinterval_i = \time_{i+1} - \time_i$ such that $0 = \time_0 < \time_1 < \dots < \time_{\stepcount} = \totaltimeinterval$. $\mpower = \{ \vpower_{\time_i} \}_{i \in \timeindex}$ is a set of power vectors $\vpower_i$ that correspond to the time intervals $\timeinterval_i$ indexed by $\timeindex = \{ 0, \dots, \stepcount - 1 \}$.

A temperature profile $\temperatureprofile$ with respect to a power profile $\powerprofile$ is defined as a tuple $\temperatureprofile = (\timepartition, \mtemperature)$, where the time partition $\timepartition$ is the same as for $\powerprofile$ and $\mtemperature = \{ \vtemperature_{\time_i} \}_{i \in \timeindex}$ is a set of the corresponding temperature vectors for each of the time intervals.

\subsection{Stochastic Process}
Let $\pspace$ be a probability space \cite{oksendal2003}. $\outcomeset$ is a set of outcomes, $\salgebra$ is a $\sigma$-algebra on $\outcomeset$, and $\pmeasure$ is a probability measure. A $\salgebra$-feasible function $X: \outcome \to \real$ is called a random variable. $X \sim \normal(\mean{X}, \deviation{X}^2)$ denotes a normal distribution, where $\mean{X}$ and $\deviation{X}^2$ are the mean and variance, respectively. A vector of random variables $\v{X}$ is said to have a multivariate normal distribution if any linear combination of its component is a (univariate) normal variable, this fact is denoted as $\v{X} \sim \normal(\expectation{\v{X}}, \covariance{\v{X}})$, where $\expectation{\v{X}}$ and $\covariance{\v{X}}$ are the mean vector and covariance matrix, respectively.

A stochastic process is a parametrized collection of random variables $\{ X_\time \}_{\time \in \timeset}$, where $\timeset$ is the parameter space, the half line $[0, \infty)$ meaning time. $\{ \v{X}_\time \}_{\time \in \timeset}$ denotes a multidimensional stochastic process.
