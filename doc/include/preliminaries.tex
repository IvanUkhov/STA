\subsection{Architecture Model}
Let $\platform = \{ \core_i: i = 0, \dots, \corecount - 1 \}$ be a multiprocessor platform modeled as a set of processing elements referenced to as cores. Let $\physics$ be a set of physical parameters of the platform and ambience. $\physics$ include the floorplan of the chip, geometry of the die and thermal package, thermal characteristics the ambience and materials of the die and package.

\subsection{Power and Thermal Models}
Given $\physics$, an equivalent thermal RC circuit of the system is constructed \cite{kreith2000}. The circuit is composed of a set of thermal nodes denoted by $\nodeset = \{ \node_i: i = 0, \dots, \nodecount - 1 \}$ is an index set. Each processing element is mapped onto a set of thermal nodes. The nodes that belong to processing elements are called active, otherwise, passive.

The dynamic power dissipation of the system is denoted by a vector $\vpower = (\power_0, \dots, \power_{\nodecount - 1})^T$, where $\power_i$ is the power of the $i$th node. Passive nodes dissipate no power. The corresponding temperature variation is denoted by $\vtemperature = (\temperature_0, \dots, \temperature_{\nodecount - 1})^T$, where $\temperature_i$ is the temperature of the $i$th node.

The thermal behaviour of the system is modeled with the following heat equation:
\begin{equation} \equlabel{fourier}
  \mcapacitance \frac{d\vtemperature(\time)}{d\time} + \mconductance (\vtemperature(\time) - \vtemperature_{amb}) = \vpower(\time)
\end{equation}
where $\mcapacitance$ and $\mconductance$ are $\nodecount \times \nodecount$ matrices of the thermal capacitance and conductance, respectively. $\vtemperature_{amb}$ is the ambient temperature. Without loss of generality, let $\vtemperature_{amb} = 0$.

A discrete dynamic power profile of the system during a time interval $\totaltimeinterval$ is defined as a tuple $\powerprofile = (\timepartition, \mpower)$. $\timepartition = \{ \time_i: i = 0, \dots, \stepcount \}$ is a partition of $\totaltimeinterval$ into $\stepcount$ subintervals $\timeinterval_i = \time_{i+1} - \time_i$ such that $0 = \time_0 < \time_1 < \dots < \time_{\stepcount} = \totaltimeinterval$. $\mpower = \{ \vpower_i: i = 0, \dots, \stepcount - 1 \}$ is a set of power vectors $\vpower_i$ that correspond to the time intervals $\timeinterval_i$.

A temperature profile $\temperatureprofile$ with respect to a power profile $\powerprofile$ is defined as a tuple $\temperatureprofile = (\timepartition, \mtemperature)$, where the time partition $\timepartition$ is the same as for $\powerprofile$ and $\mtemperature = \{ \vtemperature_i: i = 0, \dots, \stepcount - 1 \}$ is a set of the corresponding temperature vectors for each of the time intervals.

\subsection{Stochastic Process}
Let $\pspace$ be a probability space \cite{oksendal2003}. $\outcomeset$ is a set of outcomes, $\salgebra$ is a $\sigma$-algebra on $\outcomeset$, and $\pmeasure$ is a probability measure. A $\salgebra$-feasible function $X: \outcome \to \real$ is called a random variable. $X \sim \normal(\mean{X}, \variance{X})$ denotes a normal distribution, where $\mean{X}$ and $\variance{X} = \deviation{X}^2$ are the mean and variance, respectively. A vector of random variables $\v{X}$ is said to have a multivariate normal distribution if any linear combination of its component is a (univariate) normal variable, this fact is denoted as $\v{X} \sim \normal(\mean{\v{X}}, \covariance{\v{X}})$, where $\mean{\v{X}}$ and $\covariance{\v{X}}$ are the mean vector and covariance matrix, respectively.

A stochastic process is a parametrized collection of random variables $\{ X_\time \}_{\time \in \timeset}$, where $\timeset$ is the parameter space, the half line $[0, \infty)$ meaning time. $\{ \v{X}_\time \}_{\time \in \timeset}$ denotes a multidimensional stochastic process.
