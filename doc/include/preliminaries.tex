\subsection{Architecture and Application Models}
Consider a multiprocessor system $\system = (\platform, \application)$. $\platform = \{ \core_i \}_{i \in \coreindex}$ is a platform modeled as a set of processing elements, referenced to as cores, with an index set $\coreindex = \{ 1, \dots, \corecount \}$. $\application = (\taskset, \edgeset, \period)$ is an application modeled as a direct acyclic task graph, where $\taskset = \{\task_i\}_{i \in \taskindex}$ is the set of tasks with an index set $\taskindex = \{ 1, \dots, \taskcount \}$, $\edgeset$ is a set of data dependencies between tasks, $\period$ is the period. A mapping of the application onto the given platform is a function $\mapping: \taskindex \to \coreindex$ that assigns a core to each task. $\taskindex^{\core_i} = \{ k: k \in \taskindex, \mapping(k) = i \}$ denotes the index set of the tasks mapped onto $\core_i \in \coreset$. A schedule of the application is a function $\schedule: \taskindex \to \positivereal$ that determines the start time of each task.

\subsection{Power Model}
The dynamic power dissipation of a task $\task_j \in \taskset$ running on a core $\core_i \in \coreset$ is denoted by $U^{\core_i}_{\task_j}$; when $\mapping$ is fixed, the superscript is dropped. The power dissipation does not change during the whole execution of the task, otherwise, the task can be split into a sequence of tasks where the assumption holds.

\subsection{Thermal Model}
Physical characteristics of the system are denoted by $\physics$. $\physics$ include the floorplan of the chip, geometry of the die and thermal package, thermal characteristics of the ambience and materials of the die and package.

Given $\physics$, an equivalent thermal RC circuit of the system is constructed \cite{kreith2000}. The circuit is composed of a set of thermal nodes denoted by $\nodeset = \{ \node_i \}_{i \in \nodeindex}$, where $\nodeindex = \{ 1, \dots, \nodecount \}$ is an index set. Nodes that belong to processing elements are called active, otherwise, passive. $\nodeindexactive$ and $\nodeindexpassive$ are the corresponding index sets. The dynamic power dissipation of a node $\node_i \in \nodeset$ is denoted by $\power_{\node_i}$. Passive nodes dissipate no power, i.e., $\power_{\node_i} = 0, \forall i \in \nodeindexpassive$. Each processing element is mapped onto one thermal node\footnote{If it is not the case, processing elements in the floorplan are partitioned into several blocks, each for one thermal node, and their dynamic power dissipation is split proportionally to the area of the blocks.}; the injective function $\circuit: \coreindex \injection \nodeindex$ performs the mapping.

The temperature behaviour of thermal nodes is modeled with the following heat equation:
\begin{equation} \equlabel{fourier}
  \mcapacitance \frac{d\vtemperature(\time)}{d\time} + \mconductance (\vtemperature(\time) - \vtemperature_{amb}) = \vpower(\time)
\end{equation}
where $\vtemperature$ and $\vpower$ are $\nodecount$-vectors of temperature and dynamic power, respectively. $\mcapacitance$ and $\mconductance$ are $\nodecount \times \nodecount$ matrices of the thermal capacitance and conductance, respectively. $\mcapacitance$ is diagonal, $\mconductance$ is symmetric. $\vtemperature_{amb}$ is the ambient temperature; without loss of generality, let $\vtemperature_{amb} = 0$.

\subsection{Stochastic Process}
Let $\pspace$ be a probability space. $\outcomeset$ is a set of outcomes, $\salgebra$ is a $\sigma$-algebra on $\outcomeset$, and $\pmeasure$ is a probability measure. A $\salgebra$-feasible function $\s{X}: \outcome \to \real$ is called a random variable. A stochastic process is a parametrized collection of random variables $\{ \s{X}_\time \}_{\time \in \timeset}$, where $\timeset$ is the parameter space, the half line $[0, \infty)$ meaning time.

The power dissipation and, consequently, temperature behaviour are stochastic processes. \equref{fourier} takes the following stochastic form:
\begin{equation} \equlabel{stochastic-fourier}
  \mcapacitance \frac{d\svtemperature(\time, \outcome)}{d\time} + \mconductance (\svtemperature(\time, \outcome) - \vtemperature_{amb}) = \svpower(\time, \outcome)
\end{equation}
