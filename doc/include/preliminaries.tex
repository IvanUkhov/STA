\subsection{Architecture Model}
Consider a \definition{multiprocessor system} $\system = (\platform, \physics)$. $\platform = \{ \core_i \}_{i \in \coreindex}$ is a joint set of programmable processors and communication buses with a single index set $\coreindex = \{0, \dots, \corecount - 1\}$. Without loss of generality, we do not treat processors and buses separately and shall refer to them as \definition{cores}. $\physics$ denotes a set of \definition{physical characteristics} of the platform and ambience, which includes the floorplan of the cores, geometry of the die and thermal package, thermal parameters of the ambience and materials that the die and package are made of.

\subsection{Power and Thermal Models}
Given $\system$, an \definition{equivalent thermal RC circuit} of the system is constructed \cite{kreith2000}. The circuit is composed of a set of \definition{thermal nodes} denoted by $\nodeset = \{ \node_i \}_{i \in \nodeindex}$, where $\nodeindex = \{ 0, \dots, \nodecount - 1 \}$ is an index set. The structure of the circuit depends on the desired level of details, i.e., accuracy. If a node belongs to a core, it is called \definition{active}, otherwise, \definition{passive}.

The dynamic power dissipation of the constructed circuit at time $\time$ is denoted by a column vector $\vpower(\time) = \vector{\power_i(\time)}_{i \in \nodeindex}$ (hereafter this notation is used to emphasize elements of a vector), where $\power_i(\time)$ is the power of the $i$th node. Passive nodes dissipate no power. The corresponding temperature variation is denoted by $\vtemperature(\time) = \vector{\temperature_i(\time)}_{i \in \nodeindex}$, where $\temperature_i(\time)$ is the temperature of the $i$th thermal node.

The thermal behaviour of the system is modeled with the following heat equation:
\begin{equation} \equlabel{fourier}
  \mcapacitance \frac{d\vtemperature(\time)}{d\time} + \mconductance (\vtemperature(\time) - \vtemperature_{amb}) = \vpower(\time)
\end{equation}
where $\mcapacitance$ and $\mconductance$ are $\nodecount \times \nodecount$ matrices of the thermal capacitance and conductance, respectively. $\vtemperature_{amb}$ is the ambient temperature. Without loss of generality, let $\vtemperature_{amb} = 0$.

A discrete \definition{dynamic power profile} of the system over a time interval $\period$ is defined as a tuple $\powerprofile = (\timepartition, \mpower)$. $\timepartition = \{ \timeinterval_i \}_{i \in \timeindex}$ is a partition of $\period$ into $\stepcount$ subintervals, indexed by $\timeindex = \{ 0, \dots, \stepcount - 1 \}$, such that $\sum_{i \in \timeindex} \timeinterval_i = \period$. $\mpower = \matrix{\power_{ij}}_{i \in \nodeindex, j \in \timeindex}$ is a $\nodecount \times \stepcount$ matrix of the dynamic power dissipation of each thermal node $\node_i$ in each time interval $\timeinterval_j$. $\vpower_j = \vector{\power_{ij}}_{i \in \nodeindex}$ denotes the $j$th column of $\mpower$, i.e., the vector of the dynamic power dissipation of all nodes in the $j$th time interval.

A discrete \definition{temperature profile} $\temperatureprofile$ with respect to a power profile $\powerprofile$ is defined as a tuple $\temperatureprofile = (\timepartition, \mtemperature)$, where the time partition $\timepartition$ remains the same as for $\powerprofile$ and $\mtemperature = \matrix{\vtemperature_{ij}}_{i \in \nodeindex, j \in \timeindex}$ is a matrix of the corresponding temperature values for each thermal node $\node_i$ in each time interval $\timeinterval_j$. $\vtemperature_j = \vector{\temperature_{ij}}_{i \in \nodeindex}$ denotes the $j$th column of $\mtemperature$, i.e., the vector of temperature values of all nodes in the $j$th time interval.

\subsection{Stochastic Process}
Let $\pspace$ be a \definition{probability space} \cite{oksendal2003}, where $\outcomeset$ is a set of outcomes, $\salgebra$ is a $\sigma$-algebra on $\outcomeset$, and $\pmeasure$ is a probability measure.

A $\salgebra$-measurable function $X: \outcome \to \real$ is called a \definition{\rv}. $X \sim \normal(\mean{X}, \deviation{X}^2)$ denotes a \definition{normally distributed \rv}, where $\mean{X}$ and $\deviation{X}^2$ are the \definition{mean} and \definition{variance}, respectively. A vector of \rvs\ $\v{X}: \outcome \to \real^n$ is called a \definition{\mrv}. $\v{X} \sim \normal(\expectation{\v{X}}, \covariance{\v{X}})$ denotes a \definition{normally distributed \mrv}, where $\expectation{\v{X}}$ and $\covariance{\v{X}}$ are the \definition{mean vector} and \definition{covariance matrix}, respectively.

A \definition{stochastic process} is a parametrized collection of \rvs\ $\{ X_\time \}_{\time \in \timeset}$, where $\timeset$ is the parameter space, the half line $[0, \infty)$ meaning time. $\{ \v{X}_\time \}_{\time \in \timeset}$ denotes a \definition{multidimensional stochastic process}.

\subsection{Process Variation} \seclabel{process-variation}
Due to the process variation, the actual power consumption of the system can deviate from the nominal values. Therefore, the dynamic power profile of the system is assumed to be uncertain, i.e., the power dissipation $\power_{ij}$ of the $i$th core in the $j$th time interval is a \rv.

Assume that the uncertainties, introduced by the process variation into a chip, do not depend on time, i.e., a core within a chip, performing a curtain operation, always exhibits the same deviation from the nominal value. Also, due to the nature of the process variation, the direction of the deviation of a core is fixed, i.e., a core can always be either \definition{cold}, meaning that its power always tends to be lower than the nominal value, or \definition{hot}, meaning that its power always tends to be higher than the nominal value.
