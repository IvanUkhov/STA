\subsection{Architecture and Application Models}
Consider a multiprocessor system $\system = (\platform, \application)$. $\platform = \{ \core_i \}_{i \in \coreindex}$ is a platform modeled as a set of processing elements, referenced to as cores, with an index set $\coreindex = \{ 0, \dots, \corecount - 1 \}$. $\application = (\taskset, \edgeset)$ is an application modeled as a direct acyclic task graph, where $\taskset = \{\task_i\}_{i \in \taskindex}$ is the set of tasks with an index set $\taskindex = \{ 0, \dots, \taskcount - 1 \}$, $\edgeset$ is a set of data dependencies between tasks. A mapping of the application onto the given platform is a function $\mapping: \taskindex \to \coreindex$ that assigns a core to each task. $\taskindex^{\core_i} = \{ k: k \in \taskindex, \mapping(k) = i \}$ denotes the index set of the tasks mapped onto $\core_i \in \coreset$. A schedule of the application is a function $\schedule: \taskindex \to \positivereal$ that determines the start time of each task.

\subsection{Power and Thermal Models}
A core $\core_i \in \coreset$ running a task $\task_j \in \taskset$ dissipates dynamic power $U^{\core_i}_{\task_j}$. The power dissipation does not change during the whole execution of the task. If it is not the case, the task can be split into a sequence of tasks where the assumption holds.

The physical characteristics of the platform and ambience are denoted as $\physics$. $\physics$ include the floorplan of the chip, geometry of the die and thermal package, thermal parameters of the materials, the die and package are made of, and ambience.

Given $\platform$ and the corresponding $\physics$, a compact thermal model is constructed \cite{huang2006}. The Fourier thermal model is employed, where the heat equation has the following matrix form:
\begin{equation} \equlabel{fourier}
  \mcapacitance \frac{d\vtemperature(\time)}{d\time} + \mconductance (\vtemperature(\time) - \vtemperature_{amb}) = \vpower(\time)
\end{equation}
where $\vtemperature$ and $\vpower$ are $\nodecount$-vectors of temperature and dynamic power, respectively. $\mcapacitance$ and $\mconductance$ are $\nodecount \times \nodecount$ matrices of the thermal capacitance and conductance, respectively. $\mcapacitance$ is diagonal, $\mconductance$ is symmetric. $\vtemperature_{amb}$ is the ambient temperature; without loss of generality, let $\vtemperature_{amb} = 0$.
