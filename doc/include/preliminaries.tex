\subsection{Architecture Model}
Consider a multiprocessor system that is composed of a number of processing elements and a thermal package. Denote $\system$ the set of parameters that describes the system and includes the following information:
\begin{itemize}
  \item The floorplan of the die (the location and dimensions of the processing elements).
  \item The configuration of the thermal package (the dimensions of each of the layers).
  \item The thermal parameters of the materials that the die and package are made of (the thermal conductivity, specific heat, convection capacitance, and convection resistance).
  \item The temperature of the ambience.
\end{itemize}

\subsection{Power and Thermal Models} \seclabel{thermal-model} \seclabel{power-model}
Given $\system$, an equivalent thermal RC circuit of the system is constructed \cite{kreith2000}. The circuit is composed of $\nodecount$ \definition{thermal nodes} denoted by a set $\nodeset = \{ \node_i \}_{i \in \nodeindex}$, where $\nodeindex = \{ 0, \dots, \nodecount - 1 \}$ is the corresponding index set. The structure and number of thermal nodes depend on the desired level of details, defining the accuracy of the final model.

The total power dissipation of the constructed circuit at time $\time$ is denoted by a column vector $\vpower(\time) = \vector{\power_i(\time)}_{i \in \nodeindex}$ (hereafter this notation is used to emphasize elements of a vector), where $\power_i(\time)$ is the power of the $i$th thermal node\footnote{The passive nodes, i.e., the nodes that belong to the thermal package, dissipate no power.}. The total power is modeled as the sum of the dynamic power $\vpower_\dyn(\time)$ and leakage power $\vpower_\leak(\time)$, which we shall further discuss in \secref{process-variation}. The corresponding temperature is denoted by a column vector $\vtemp(\time) = \vector{\temp_i(\time)}_{i \in \nodeindex}$, where $\temp_i(\time)$ is the temperature of the $i$th thermal node.

The thermal behaviour of the system is modeled using the following heat equation:
\begin{equation} \equlabel{fourier}
  \mcapacitance \frac{d\vtemp(\time)}{d\time} + \mconductance (\vtemp(\time) - \vtemp_\amb) = \vpower(\time)
\end{equation}
where $\mcapacitance$ and $\mconductance$ are $\nodecount \times \nodecount$ matrices of the thermal capacitance and conductance, respectively, $\vtemp_\amb$ is the temperature of the ambience.

A (discrete) total \definition{power profile} of the system over a time interval $\period$ is defined as a tuple $\pprofile{}$, where $\timepartition = \{ 0 = \time_0 < \dots < \time_{\stepcount} = \period \}$ is a partition of $\period$ into $\stepcount$ subintervals $\timeinterval_j = \time_{j+1} - \time_j$, indexed by $\timeindex = \{ 0, \dots, \stepcount - 1 \}$, and $\mpower = \matrix{\power_{ij}}_{i \in \nodeindex, j \in \timeindex}$ is a $\nodecount \times \stepcount$ matrix (hereafter this notation is used to emphasize elements of a matrix) of the power dissipation of each thermal node $\node_i$ in each time interval $\timeinterval_j$. $\vpower_j = \vector{\power_{ij}}_{i \in \nodeindex}$ denotes the $j$th column of $\mpower$, i.e., the vector of the power dissipation of all nodes in the $j$th time interval. Since the total power is modeled as the sum of the dynamic power and leakage power, we shall use $\pprofile{\dyn}$ and $\pprofile{\leak}$ to distinguish between these two parts of $\pprofile{}$.

A (discrete) \definition{temperature profile} with respect to a power profile $\pprofile{}$ is defined as a tuple $\tprofile{}$, where the time partition $\timepartition$ remains the same as for $\pprofile{}$ and $\mtemp = \matrix{\vtemp_{ij}}_{i \in \nodeindex, j \in \timeindex}$ is a matrix of the corresponding temperature values for each thermal node $\node_i$ in each time interval $\timeinterval_j$. $\vtemp_j = \vector{\temp_{ij}}_{i \in \nodeindex}$ denotes the $j$th column of $\mtemp$, i.e., the vector of temperature values of all nodes in the $j$th time interval.

\subsection{Elements of Probability Theory} \seclabel{probability-theory}
Let $\pspace$ be a probability space \cite{durrett2010}, where $\outcomeset$ is a set of outcomes, $\salgebra$ is a $\sigma$-algebra on $\outcomeset$, and $\pmeasure$ is a probability measure. A $\salgebra$-measurable function $X: \outcome \to \real$ is called a \definition{\rv}. $\mean{X}$ and $\dev{X}^2$ represent the mean and variance of $X$, respectively. $X \sim \normal(\mean{X}, \dev{X}^2)$ denotes a \rv\ with a normally distributed \rv.

A vector of \rvs\ $\v{X}: \outcome \to \real^n$ is called a \definition{\mrv}. $\expect{\v{X}}$ and $\cov{\v{X}} = \expect{(\v{X} - \expect{\v{X}})(\v{X} - \expect{\v{X}})^T}$ represent the mean vector and covariance matrix of $\v{X}$, respectively. A normalized version of the covariance matrix is known as the correlation matrix, where $(i,j)$th element of the former is divided by $\dev{X_i} \dev{X_j}$, loosing the units of measure, but preserving the degree of linear dependencies in terms of the Person correlation coefficient\footnote{The Pearson correlation coefficient takes values from $-1$ to $1$ and is defined by $\rho_{ij} = \expect{(X_i - \mean{X_i})(X_j - \mean{X_j})}/(\dev{X_i}\dev{X_j})$.}. $\v{X} \sim \normal(\expect{\v{X}}, \cov{\v{X}})$ denotes a normally distributed \mrv.

A \definition{\sp} is a parametrized collection of \rvs\ $\{ X(\time) \}_{\time \in \timeset}$, where $\timeset$ is the parameter space, the half line $[0, \infty)$ meaning time. $\{ \v{X}(\time) \}_{\time \in \timeset}$ denotes a multidimensional \sp.

As any real symmetric matrix, a covariance matrix $\cov{\v{X}}$ (hence, the correlation matrix as well) can be factorized using the eigenvalue decomposition \cite{press2007} as $\cov{\v{X}} = \m{U} \m{V} \m{U}^T$, where $\m{U}$ and $\m{V}$ are an orthogonal matrix of the eigenvectors and a diagonal matrix of the eigenvalues of $\cov{\v{X}}$, respectively. Denote $\factorize{\cov{\v{X}}} = \m{U} \m{V}^{1/2}$. Therefore, a \mnrv\ $\v{X} \sim \normal(\vzero, \cov{\v{X}})$ has the same distribution as a \msnrv\ $\v{Y} \sim \normal(\vzero, \mone)$, multiplied by $\factorize{\cov{\v{X}}}$ on the left.

Also, throughout the paper, we shall use the following property of \mnrvs. A linear combination $\m{A} \v{X} + \m{B} \v{Y}$ of two dependent \mnrvs\ $\v{X} \sim \normal(\expect{\v{X}}, \cov{\v{X}})$ and $\v{Y} \sim \normal(\expect{\v{X}}, \cov{\v{X}})$ is a \mnrv\ $\v{Z} \sim \normal(\m{A} \expect{\v{X}} + \m{B} \expect{\v{X}}, \dbl{\m{A}, \cov{\v{X}}} + \dbl{\m{B}, \cov{\v{Y}}} + \dbl{\dbl{\m{A}, \cov{\v{X}, \v{Y}}, \m{B}}})$, where $\dbl{\m{C}, \m{D}} = \m{C} \m{D} \m{C}^T$, $\dbl{\dbl{\m{C}, \m{D}, \m{E}}} = \m{C} \m{D} \m{E}^T + \m{E} \m{D}^T \m{C}^T$, and $\cov{\v{X}, \v{Y}}$ is the cross-covariance matrix of the variables $\v{X}$ and $\v{Y}$. If the variables are independent, the cross-covariance matrix is zero.

\subsection{Process Variation and Environmental Noise} \seclabel{uncertainties}
In this paper, we aim to model two sources of uncertainties: the process variation, discussed in \secref{process-variation}, and environmental noise, discussed in \secref{noise}. Both components affect the dynamic power dissipation $\power_\dyn$ as well as the leakage power $\power_\leak$. Consequently, the actual power consumption of the system can deviate from the nominal values, becoming a \rv. $\power_\leak$ was shown to be considerably more sensible to the process variation than $\power_\dyn$. However, when $\power_\leak$ is negligible small, the variations in $\power_\dyn$ play the key role. In this paper, we take into consideration variations in both components.
