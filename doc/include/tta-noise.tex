The solution given by \equref{solution-full} is applicable for one time interval with constant power. In order to model the whole duration $\period$ of the power profile, the computations should be performed for each of the subintervals sequentially. Therefore, we have the following recurrent expression:
\begin{equation} \equlabel{tta-fourier-recurrence-noise}
  \vtemp_{i + 1} = \m{A}_i \vtemp_i + \m{B}_i \vmean{\dyn \; i} + \m{D}_i, \sep i \in \timeindex
\end{equation}
where $\m{D}_i = \m{D}(\timeinterval_i)$ according to \equref{d}.

Taking into consideration the properties of the Wiener process, $\vtemp_i$ is independent of $\m{D}_i$. Therefore, $\vtemp_{i+1}$ is a \mnrv:
\begin{align*}
  & \vtemp_{i + 1} \sim \normal(\expect{\vtemp_{i + 1}}, \cov{\vtemp_{i + 1}}), \sep i \in \timeindex \\
  & \expect{\vtemp_{i+1}} = \m{A}_i \expect{\vtemp_i} + \m{B}_i \vmean{\dyn \; i} \\
  & \cov{\vtemp_{i+1}} = \dbl{\m{A}_i, \cov{\vtemp_i}} + \cov{\m{D}_i}
\end{align*}

Alternatively, non-recurrent expressions can be obtained by performing an iterative repetition of \equref{tta-fourier-recurrence-noise}:
\begin{equation} \equlabel{tta-fourier-noise-non-recurrent}
  \vtemp_{i + 1} = \am{A}_{i0} \vtemp_0 + \m{F}_i + \v{L}_i, \sep i \in \timeindex
\end{equation}
where $\m{F}_i$ is computed according to \equref{f} with the only difference that $\vmean{i} \equiv \vmean{\dyn \; i}$ and:
\begin{equation} \equlabel{l}
  \v{L}_i = \sum_{j = 0}^{i} \am{A}_{i(j+1)} \m{D}_j
\end{equation}
Therefore:
\begin{align*}
  & \vtemp_{i + 1} \sim \normal(\expect{\vtemp_{i + 1}}, \cov{\vtemp_{i + 1}}), \sep i \in \timeindex \\
  & \expect{\vtemp_{i + 1}} = \am{A}_{i0} \expect{\vtemp_0} + \m{F}_i \\
  & \cov{\vtemp_{i + 1}} = \dbl{\am{A}_{i0}, \cov{\vtemp_0}} + \m{M}_i
\end{align*}
where, for convenience, we denote:
\begin{equation} \equlabel{m}
  \m{M}_i = \sum_{j = 0}^{i} \dbl{\am{A}_{i(j+1)}, \cov{\m{D}_j}}
\end{equation}
