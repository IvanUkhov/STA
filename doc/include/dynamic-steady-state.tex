As in the SSSTA (\secref{sssta}), the given power profile is assumed to be periodic. However, instead of aiming at a hypotherical thermal balance with a single constant temperature for each of the thermal nodes, the \definition{dynamic steady-state temperature analysis} (DSSTA) delivers a set of periodic curves, similar to the TTA (\secref{tta}), that describe a periodic pattern of the thermal behaviour of the system.

\subsection{Problem Formulation}
The sources of uncertainties are a random noise and the process variation.

Given:
\begin{itemize}
  \item A multiprocessor system $\system = (\platform, \physics)$.
  \item A nominal dynamic power profile $\powerprofile$.
  \item Knowledge about uncertainties presented in the system (specified in the following subsections).
\end{itemize}

Find:
\begin{itemize}
  \item The probability distribution of the periodic temperature profile $\temperatureprofile$ with respect to $\powerprofile$.
\end{itemize}

\subsection{Solution with Noise}
Additional assumptions and givens are similar to those in \secref{ta-noise}.

\subsection{Solution with Process Variation}
Additional assumptions and givens are similar to those in \secref{ta-process-variation}. Since the resulting temperature curves should be periodic, the following boundary condition is to be satisfied:
\begin{equation} \equlabel{dss-boundary}
  \vtemperature_0 = \vtemperature_{\stepcount}
\end{equation}
Using \equref{ta-fourier-non-recurrent}, the temperature vector at the end of the curves can be computed as:
\begin{align*}
  \vtemperature_{\stepcount} & = \am{A}_{\stepcount - 1} \vtemperature_0 + \m{F}_{\stepcount - 1} + \m{H}_{\stepcount - 1} \vstdnormal
\end{align*}
Taking into consideration the boundary condition given by \equref{dss-boundary}, the solution of the last equation with respect to $\vtemperature_0$ is the following:
\[
\vtemperature_0 = (\mone - \am{A}_{\stepcount - 1})^{-1} (\m{F}_{\stepcount - 1} + \m{H}_{\stepcount - 1} \vstdnormal)
\]
Consequently, the mean and covariance at the beginning of the periodic power profile are:
\begin{align*}
  & \vtemperature_0 \sim \normal(\expectation{\vtemperature_0}, \covariance{\vtemperature_0}) \\
  & \expectation{\vtemperature_0} = (\mone - \am{A}_{\stepcount - 1})^{-1} \v{F}_{\stepcount - 1} \\
  & \covariance{\vtemperature_0} = (\mone - \am{A}_{\stepcount - 1})^{-1} \v{H}_{\stepcount - 1} \v{H}_{\stepcount - 1}^T (\mone - \am{A}_{\stepcount - 1}^T)^{-1}
\end{align*}
The rest of the temperature vectors is successively found using \equref{ta-fourier-recurrence}, hence:
\begin{align*}
  & \vtemperature_{i + 1} \sim \normal(\expectation{\vtemperature_{i + 1}}, \covariance{\vtemperature_{i + 1}}), \sep i \in \timeindex \setminus \{\stepcount - 1\} \\
  & \expectation{\vtemperature_{i + 1}} = \m{A}_i \expectation{\vtemperature_i} + \m{B}_i \vmean{i} \\
  & \covariance{\vtemperature_{i + 1}} = \m{A}_i \covariance{\vtemperature_i} \m{A}_i^T + \m{B}_i \ddeviation{i}^2 \m{B}_i^T \\
  & {} \qquad \qquad + \m{A}_i \covariance{\vtemperature_i, \vpower_i} \m{B}_i^T + \m{B}_i \covariance{\vtemperature_i, \vpower_i}^T \m{A}_i^T
\end{align*}
