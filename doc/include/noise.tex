Suppose, the system at each moment of time is exposed to a random noise, denoted by $\noise(\time)$, which affects the total power dissipation and, consequently, the resulting temperature. In this paper, we assume that $\noise(\time)$ is the white noise with a known covariance matrix $\cov{\noise}$.

Due to the assumptions, $\noise(\time)$ is a \mnrv\ at each moment of time $\time$, i.e., $\noise(\time) \sim \normal(\vzero, \cov{\noise})$. We decompose the noise as $\noise(\time) = \ddev{\noisy} \vnormal_\noisy(\time)$, where $\ddev{\noisy} = \factorize{\cov{\noise}}$ and the stochastic process $\{ \vnormal_\noisy(\time) \}_{\time \in \timeset}$ is the standard $\nodecount$-dimensional white noise \cite{oksendal2003}, i.e., a set of vectors of independent \snrvs\ $\vnormal_\noisy(\time) \sim \normal(\vzero, \mone)$. Hence, using \equref{fourier-constant-power}, we have the following stochastic process within one time interval of constant power:
\begin{equation} \equlabel{fourier-noise}
  \mcapacitance \frac{d\vtemp(\time)}{d\time} + \mconductance \vtemp(\time) = \vpower + \ddev{\noisy} \vnormal_\noisy(\time)
\end{equation}
\equref{fourier-noise} can be rewritten in the differential form:
\begin{equation} \equlabel{fourier-wiener}
  d\vtemp(\time) = \mcapacitance^{-1} (\vpower - \mconductance \vtemp(\time))d\time + \mcapacitance^{-1} \ddev{\noisy} d\m{W}(\time)
\end{equation}
where $d\m{W}(\time) = \vnormal_\noisy(\time) d\time$ with $\{ \m{W}(\time) \}_{\time \in \timeset}$ being the Wiener process or Brownian motion \cite{oksendal2003}. The integration of \equref{fourier-wiener} cannot be done in the regular Riemann-Stieltjes sense, since the Brownian motion is nowhere differentiable, therefore, a special calculus should be applied. We follow the It\^{o} interpretation \cite{oksendal2003} of the stochastic differential equation in \equref{fourier-wiener}. It can be seen that the stochastic process $\{ \vtemp(\time) \}_{\time \in \timeset}$ is a multidimensional Ornstein-Uhlenbeck process \cite{kloeden1992} driven by the Brownian motion. In order to solve \equref{fourier-wiener}, we apply the It\^{o} formula \cite{oksendal2003} to the function $\ecg{}{t} \vtemp(\time)$ and obtain:
\begin{align*}
  d(\ecgt \vtemp(\time)) & = \cg \ecgt \vtemp(\time) d\time + \ecgt d\vtemp(\time) \\
  & = \ecgt \mcapacitance^{-1} (\vpower d\time + \ddev{\noisy} d\v{W}(\time))
\end{align*}
The solution can be found by taking integrals on both sides:
\begin{align*}
  \vtemp(\time) = & \emcgt{} \vtemp(0) - \\
    & (\cg)^{-1}(\emcgt{} - \mone) \mcapacitance^{-1} \vpower + \\
    & \int_0^\time \ecg{}{(s - \time)} \mcapacitance^{-1} \ddev{\noisy} \; d\m{W}(s)
\end{align*}
To simplify the further derivation, we introduce the following notation:
\begin{equation} \equlabel{d}
  \v{D}(\time) = \int_0^{\time} \ecg{}{(s - \time)} \mcapacitance^{-1} \ddev{\noisy} \: d\m{W}(s)
\end{equation}
Therefore, taking into consideration \equref{a} and \equref{b}, \equref{solution-full} can be rewritten:
\begin{equation} \equlabel{solution-full}
  \vtemp(\time) = \m{A}(\time) \vtemp(0) + \m{B}(\time) \vpower + \v{D}(\time)
\end{equation}
Since the Wiener process has independent normally distributed increments, an integration with respect to it, i.e., $\int_0^\time f(s) dW(s)$, is a \nrv. It can also be shown that in this case the mean is zero and variance is equal to $\int_0^\time f^2(s) ds$. Hence, $\v{D}(\time) \sim \normal(\vzero, \cov{\v{D}(\time)})$, i.e., $\v{D}(\time)$ is a \mnrv\ with a zero mean vector and the following covariance matrix:
\begin{align}
  \cov{\v{D}(\time)} & = \int_0^\time \ecg{2}{(s - \time)} (\mcapacitance^{-1} \ddev{\noisy})^2 \; ds \nonumber \\
  & = (2 \cg)^{-1} (\mone - \emcgt{2}) (\mcapacitance^{-1} \ddev{\noisy})^2 \equlabel{covariance-d}
\end{align}
