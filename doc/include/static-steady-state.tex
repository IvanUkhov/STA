Suppose that a given power profile is periodic and, after a sufficiently long period of time, the system reaches a hypothetical thermal balance, where temperature does not change. This state is known as \definition{static steady state}.

\subsection{Problem Formulation}
Given:
\begin{itemize}
  \item A multiprocessor system $\system = (\platform, \physics)$.
  \item A nominal dynamic power profile $\powerprofile$.
  \item The variance $\deviation{ij}^2$ of $\power_{ij} \; \forall i \in \nodeindex, j \in \timeindex$.
\end{itemize}

Find:
\begin{itemize}
  \item The probability distribution of the temperature profile $\temperatureprofile$ with respect to $\powerprofile$, when the static steady state is reached.
\end{itemize}

\subsection{Solution} \seclabel{sss-solution}
Additional assumptions:
\begin{itemize}
  \item $\power_{ij} \sim \normal(\mean{ij}, \deviation{ij}^2)$, i.e., the power dissipation is distributed normally.
  \item $\power_{i_1 j_1}$ and $\power_{i_2 j_2}$ are independent $\forall i_1, i_2 \in \nodeindex,  i_1 \neq i_2$ and $\forall j_1, j_2 \in \timeindex$.
\end{itemize}

In the steady-state case, \equref{fourier} does not depend on time and drops the derivative term:
\begin{equation} \equlabel{steady-fourier}
  \a{\vtemperature} = \mconductance^{-1} \a{\vpower}
\end{equation}
where $\a{\vpower}$ is a random vector that represents the average power dissipation of thermal nodes, consequently, $\a{\vtemperature}$ is a random vector that corresponds to temperature. $\a{\vpower}$ is modeled with a regular equation:
\[
  \a{\vpower} = \frac{1}{\stepcount} \sum_{j \in \timeindex} \vpower_j
\]
where $\vpower_j$ are \mnrvs.

As mentioned in \secref{process-variation}, \rvs\ $\power_{ij}$ for a fixed $i$ and $\forall j$ (a row in the power matrix $\mpower$) are required to jointly deviate from the nominal (mean) values into the same direction. A straight-forward solution to model such a behaviour is to represent the vectors of the power dissipation $\vpower_j$ for each time interval $\timeinterval_j$ as linear transformations of a single \msnrv\ $\vstdnormal = \vector{\stdnormal_i}_{i \in \nodeindex} \sim \normal(\vzero, \mone)$:
\begin{equation} \equlabel{power-rv}
  \vpower_j = \vector{\mean{ij} + \deviation{ij} \stdnormal_i}_{i \in \nodeindex} = \vmean{j} + \ddeviation{j} \vstdnormal
\end{equation}
where $\vmean{j} = \vector{\mean{ij}}_{i \in \nodeindex}$, i.e., the vector of the nominal power in the $j$th time interval, and $\ddeviation{j} = \diagonal{\deviation{ij}}_{i \in \nodeindex}$, i.e., a diagonal matrix. Therefore:
\[
  \a{\vpower} = \frac{1}{\stepcount} \sum_{j \in \timeindex} \vmean{j} + \frac{1}{\stepcount} \sum_{j \in \timeindex} \ddeviation{j} \; \vstdnormal
\]
It can be seen that $\a{\vpower}$ is a \mnrv\ with independent components:
\begin{align}
  & \a{\vpower} \sim \normal(\expectation{\a{\vpower}}, \covariance{\a{\vpower}}) \nonumber \\
  & \expectation{\a{\vpower}} = \frac{1}{\stepcount} \sum_{j \in \timeindex} \vmean{j} \equlabel{steady-power-mean} \\
  & \covariance{\a{\vpower}} = \frac{1}{\stepcount^2} \left[ \sum_{j \in \timeindex} \ddeviation{j} \right]^2 \equlabel{steady-power-covariance}
\end{align}
Consequently, after the linear transformation of $\a{\mpower}$ in \equref{steady-fourier}, $\a{\mtemperature}$ also becomes a \mnrv:
\begin{align*}
  & \a{\vtemperature} \sim \normal(\expectation{\a{\vtemperature}}, \covariance{\a{\vtemperature}}) \\
  & \expectation{\a{\vtemperature}} = \mconductance^{-1} \expectation{\a{\vpower}} \\
  & \covariance{\a{\vtemperature}} = \mconductance^{-1} \covariance{\a{\vpower}} (\mconductance^{-1})^T
\end{align*}
where $\expectation{\a{\vpower}}$ and $\covariance{\a{\vpower}}$ are found using \equref{steady-power-mean} and \equref{steady-power-covariance}, respectively.
