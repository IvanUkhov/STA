Suppose the application is periodic with the period $\period$. Assume that after a sufficiently long period of time, the system reaches a thermal balance where temperature stops changing. This state is called static steady state.

\subsection{Problem Formulation}
Given:
\begin{itemize}
  \item A multiprocessor system $\system = (\platform, \application, \physics)$.
  \item The schedule $\schedule = (\mapping, \executiontime, \starttime)$ of the application.
  \item Probability distributions of the dynamic power dissipation $\power_{\task_i}$ of tasks $\task_i \in \taskset$.
\end{itemize}

Find:
\begin{itemize}
  \item Probability distributions of the temperature variation $\temperature_i$ for all $\node_i \in \nodeset$ in the static steady state of the system.
\end{itemize}

\subsection{Fixed Power Model} \seclabel{sss-fixed}
Assumptions:
\begin{itemize}
  \item $\power^{(k)}_{\task_i} = \power_{\task_i} \sim \normal(\mean{\power_{\task_i}}, \deviation{\power_{\task_i}}^2)$, i.e., the power dissipation of a task does not change between actuations and it is a single normally distributed random variable.
  \item $\power_{\task_i}$ are independent for $\forall i \in \taskindex$.
\end{itemize}

In the steady-state case, \equref{fourier} does not depend on time and drops the derivative term:
\begin{equation} \equlabel{steady-fourier}
  \vtemperature = \mconductance^{-1} \vpower
\end{equation}
where $\vpower$ is a random vector that represents the average power dissipation of thermal nodes, consequently, $\vtemperature$ is a random vector that corresponds to temperature.

Since the power dissipation of a task does not change between actuations, it is sufficient to analyse just one period of the application. The average power dissipation of active thermal nodes is modeled using the following equation:
\begin{equation} \equlabel{node-power}
  \power_i = \frac{1}{\period} \sum_{j \in \taskindex^{\node_i}} \executiontime_j \power_{\task_j}
\end{equation}
where $\taskindex^{\node_i} = \{ j: j \in \taskindex, \circuit(\mapping(\task_j)) = \node_i \}$ denotes the index set of tasks mapped onto the core that, in its turn, is mapped onto the thermal node $\node_i$. Due to the independence of $\power_{\task_j}$, $\vpower$ is a multivariate normal random variable with independent components:
\begin{align}
  & \vpower \sim \normal(\expectation{\vpower}, \covariance{\vpower}) \nonumber \\
  & \expectation{\vpower} = \vector{ \frac{1}{\period} \sum_{k \in \taskindex^{\node_i}} \executiontime_k \mean{\power_{\task_k}} }^T_{i \in \nodeindex} \equlabel{steady-power-mean} \\
  & \covariance{\vpower} = \diagonal{\frac{1}{\period^2} \sum_{k \in \taskindex^{\node_i}} \executiontime_k^2 \deviation{\power_{\task_k}}^2}_{i \in \nodeindex} \equlabel{steady-power-covariance}
\end{align}
where shortcuts $\vector{x_i}^T_{i \in I}$ and $\diagonal{x_i}_{i \in I}$ are used to denote a transposed vector and diagonal matrix, respectively. Consequently, after the linear transformation of $\mpower$ in \equref{steady-fourier}, $\mtemperature$ is also a multivariate normal random variable:
\begin{align*}
  & \vtemperature \sim \normal(\expectation{\vtemperature}, \covariance{\vtemperature}) \\
  & \expectation{\vtemperature} = \mconductance^{-1} \expectation{\vpower} \\
  & \covariance{\vtemperature} = \mconductance^{-1} \covariance{\vpower} (\mconductance^{-1})^T
\end{align*}
where $\expectation{\vpower}$ and $\covariance{\vpower}$ are found using \equref{steady-power-mean} and \equref{steady-power-covariance}, respectively.

\subsection{Alternating Power Model} \seclabel{sss-alternating}
Assumptions:
\begin{itemize}
  \item The power dissipation of a task over actuations of the application is a set $\{ \power^{(k)}_{\task_i} \}_{k \in \natural}$ of identically distributed random variables with means $\mean{\power_{\task_i}}$ and variances $\deviation{\power_{\task_i}}^2$.
  \item $\power^{(k)}_{\task_i}$ are independent for $\forall i \in \taskindex, \; \forall k \in \natural$.
\end{itemize}

The average power dissipation of active thermal nodes over $\periodcount$ periods of the application is modeled using the following equation:
\begin{align}
  \power_i & = \frac{1}{\periodcount \period} \sum_{j \in \taskindex^{\node_i}} \executiontime_j \sum_{k = 0}^{\periodcount - 1} \power^{(k)}_{\task_j} \nonumber \\
  & = \frac{1}{\period} \sum_{j \in \taskindex^{\node_i}} \executiontime_j \: \a{\power}_{\task_j}, \sep i \in \nodeindex \equlabel{node-average-power}
\end{align}
where $\a{\power}_{\task_j} = \frac{1}{\periodcount} \sum_{k = 0}^{\periodcount - 1} \power^{(k)}_{\task_j}$ means the average power dissipation of $\task_j$ over $\periodcount$ actuations of the application. Therefore, \equref{node-average-power} sums up all power dissipations, multiplied by the corresponding execution times, of all tasks mapped onto the specific core $\circuit^{-1}(\node_i)$ in all $\periodcount$ actuations and divides the result by the total time.

For a fixed task $\task_i \in \taskset$, $\{ \power^{(k)}_{\task_i} \}_{k \in \natural}$ is a set of independent identically distributed random variables with $\mean{\power^{(k)}_{\task_i}} = \mean{\power_{\task_i}}$ and $\deviation{\power^{(k)}_{\task_i}}^2 = \deviation{\power_{\task_i}}^2$. Hence, the conditions of the central limit theorem in the common form are satisfied. Therefore, the average power dissipation of the task $\task_i$, denoted by $\a{\power}_{\task_i}$, can be approximated with a normal distribution:
\begin{align*}
  & \a{\power}_{\task_i} \sim \normal(\mean{\a{\power}_{\task_i}}, \deviation{\a{\power}_{\task_i}}^2), \sep i \in \taskindex \\
  & \mean{\a{\power}_{\task_i}} = \mean{\power_{\task_i}} \\
  & \deviation{\a{\power}_{\task_i}}^2 = \frac{\deviation{\power_{\task_i}}^2}{\periodcount}
\end{align*}
$\a{\power}_{\task_i}$ are independent due to the first assumption. The rest is similar to \secref{sss-fixed}.

It should be noted, when modeling a periodic application that is being run for a long time, $\periodcount$ approach infinity and the variance becomes zero. In this case, the only thing that matters is the mean $\mean{\power_{\task_i}}$, which is already given. Therefore, there are no reasons of constructing this particular model.
