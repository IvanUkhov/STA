\subsection{Problem Formulation}
Given:
\begin{itemize}
  \item A multiprocessor system $\system = (\platform, \application)$.
  \item Physical characteristics the platform and ambience $\physics$.
  \item The mapping $\mapping$ and schedule $\schedule$ of the application.
  \item The execution time $\executiontime_i$ of each task $\task_i \in \taskset$ running on the corresponding core $\core_j \in \coreset, \mapping(i) = j$.
  \item The probability density function $\pdf_{\spower_{\task_i}}(\spower)$ of the power dissipation $\spower_{\task_i}(\outcome)$ of each task $\task_i \in \taskset$.
\end{itemize}

Find:
\begin{itemize}
  \item The probability density function $\pdf_{\stemperature_{\node_i}}(\stemperature)$ of the temperature $\stemperature_{\node_i}(\outcome)$ of each thermal node $\node_i \in \nodeset$ in the static steady state of the system.
\end{itemize}

Assumptions:
\begin{itemize}
  \item Normal distribution of $\spower_{\task_i}(\outcome) \sim \normal(\mean{\spower_{\task_i}}, \variance{\spower_{\task_i}})$.
  \item Independence of $\spower_{\task_i}(\outcome)$.
\end{itemize}

\subsection{Model}
In the steady-state case, \equref{stochastic-fourier} does not depend on time and drops the derivative term:
\[
  \svtemperature_\node(\outcome) = \mconductance^{-1} \svpower_\node(\outcome) = \mresistance \svpower_\node(\outcome)
\]
where $\svpower_\node(\outcome)$ is a random vector that represents the average power dissipation of thermal nodes, consequently, $\svtemperature_\node(\outcome)$ is a random vector that corresponds to temperature.

The average power dissipation of active thermal nodes is modeled using the following equation:
\[
  \spower_{\node_i}(\outcome) = \frac{1}{\period} \sum_{j \in \taskindex^{\node_i}} \executiontime_j \spower_{\task_j}(\outcome), \sep i \in \nodeindexactive
\]
Since $\spower_{\task_j}$ are normally distributed and $\spower_{\node_i}$ are linear combinations of $\spower_{\task_j}$, $\spower_{\node_i}$ are also distributed normally:
\begin{align}
  & \spower_{\node_i}(\outcome) \sim \normal(\mean{\spower_{\node_i}}, \variance{\spower_{\node_i}}), \sep i \in \nodeindex \nonumber \\
  & \mean{\spower_{\node_i}} = \frac{1}{\period} \sum_{j \in \nodeindexactive} \executiontime_j \: \mean{\spower_{\task_j}} \equlabel{power-mean} \\
  & \variance{\spower_{\node_i}} = \frac{1}{\period^2} \sum_{j \in \nodeindexactive} \executiontime_j^2 \: \variance{\spower_{\task_j}} \equlabel{power-variance}
\end{align}
The components of the temperature vector $\svtemperature_{\node}$, in their turn, are linear combinations of the components of $\svpower_{\node}$:
\[
  \stemperature_{\node_i}(\outcome) = \sum_{j \in \nodeindex} \resistance_{ij} \spower_{\node_j}(\outcome)
\]
Therefore, $\stemperature_{\node_i}$ have normal distributions:
\begin{align*}
  & \stemperature_{\node_i}(\outcome) \sim \normal(\mean{\stemperature_{\node_i}}, \variance{\stemperature_{\node_i}}), \sep i \in \nodeindex \\
  & \mean{\stemperature_{\node_i}} = \sum_{j \in \nodeindex} \resistance_{ij} \: \mean{\spower_{\node_j}} \\
  & \variance{\stemperature_{\node_i}} = \sum_{j \in \nodeindex} \resistance^2_{ij} \: \variance{\spower_{\node_j}}
\end{align*}
Using \equref{power-mean} and \equref{power-variance}:
\begin{align*}
  & \mean{\stemperature_{\node_i}} = \frac{1}{\period} \sum_{j \in \nodeindex} \resistance_{ij} \sum_{k \in \nodeindexactive} \executiontime_k \mean{\spower_{\task_k}} \\
  & \variance{\stemperature_{\node_i}} = \frac{1}{\period^2} \sum_{j \in \nodeindex} \resistance^2_{ij} \sum_{k \in \nodeindexactive} \executiontime^2_k \variance{\spower_{\node_k}}
\end{align*}
