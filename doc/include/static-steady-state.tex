Suppose that we are given a periodic dynamic power profile and are interested in the thermal behaviour of the system in the long run. In this case, it is assumed that, after a sufficiently long period of time, the system reaches such a state, where temperature either preserves a constant value or exhibits a periodic pattern. Concequently, the state is called either the \definition{static steady state} or \definition{dynamic steady state}, respectiely. Therefore, there are two types of the steady-state temperature analysis: \definition{static steady-state temperature analysis} (SSSTA), discussed in this section, and \definition{dynamic steady-state temperature analysis} (DSSTA), discussed in \secref{dssta}.

The SSSTA delivers a single vector of temperature values for each of the thermal nodes of the system.

\subsection{Problem Formulation}
The source of uncertainties is the process variation.

Given:
\begin{itemize}
  \item A multiprocessor system $\system$.
  \item A nominal dynamic power profile $\powerprofile_\dyn$.
  \item The variance $\deviation{ij}^2$ of $\power_{ij} \; \forall i \in \nodeindex, j \in \timeindex$.
\end{itemize}

Find:
\begin{itemize}
  \item The probability distribution of the temperature profile $\temperatureprofile$ with respect to $\powerprofile$, when the static steady state is reached.
\end{itemize}

\subsection{Solution} \seclabel{sss-solution}
Additional assumptions:
\begin{itemize}
  \item $\power_{ij} \sim \normal(\mean{ij}, \deviation{ij}^2)$, i.e., the power dissipation is distributed normally.
  \item $\power_{i_1 j_1}$ and $\power_{i_2 j_2}$ are independent $\forall i_1, i_2 \in \nodeindex,  i_1 \neq i_2$ and $\forall j_1, j_2 \in \timeindex$.
\end{itemize}

As discussed earlier, the system is supposed to reach a hypothetical thermal balance, where temperature does not change over the time. In this case, the power profile is averaged and given as a constant input to the thermal RC circuit. Therefore, \equref{fourier} does not depend on time and drops the derivative term:
\begin{equation} \equlabel{sss-fourier}
  \a{\vtemperature} = \mconductance^{-1} \a{\vpower}
\end{equation}
where $\a{\vpower}$ is a random vector that represents the average power dissipation of thermal nodes, consequently, $\a{\vtemperature}$ is a random vector that corresponds to the average temperature. As mentioned in \secref{thermal-model}, the ambient temperature, $\vtemperature_\amb$, should be added to the right-hand side of \equref{sss-fourier} in order to obtain the actual temperature values.

$\a{\vpower}$ is computed with an ordinal equation:
\[
  \a{\vpower} = \frac{1}{\stepcount} \sum_{j \in \timeindex} \vpower_j
\]
where $\vpower_j$ are \mnrvs. As mentioned in \secref{uncertainties}, \rvs\ $\power_{ij}$ for a fixed $i$ and $\forall j$ (a row in the power matrix $\mpower$) are required to jointly deviate from the nominal (mean) values into the same direction. Therefore:
\[
  \a{\vpower} = \frac{1}{\stepcount} \sum_{j \in \timeindex} \vmean{j} + \frac{1}{\stepcount} \sum_{j \in \timeindex} \ddeviation{j} \; \vstdnormal
\]
It can be seen that $\a{\vpower}$ is a \mnrv\ with independent components:
\begin{align}
  & \a{\vpower} \sim \normal(\expectation{\a{\vpower}}, \covariance{\a{\vpower}}) \nonumber \\
  & \expectation{\a{\vpower}} = \frac{1}{\stepcount} \sum_{j \in \timeindex} \vmean{j} \equlabel{steady-power-mean} \\
  & \covariance{\a{\vpower}} = \frac{1}{\stepcount^2} \left[ \sum_{j \in \timeindex} \ddeviation{j} \right]^2 \equlabel{steady-power-covariance}
\end{align}
Consequently, after the linear transformation of $\a{\mpower}$ in \equref{sss-fourier}, $\a{\mtemperature}$ also becomes a \mnrv:
\begin{align*}
  & \a{\vtemperature} \sim \normal(\expectation{\a{\vtemperature}}, \covariance{\a{\vtemperature}}) \\
  & \expectation{\a{\vtemperature}} = \mconductance^{-1} \expectation{\a{\vpower}} \\
  & \covariance{\a{\vtemperature}} = \mconductance^{-1} \covariance{\a{\vpower}} (\mconductance^{-1})^T
\end{align*}
where $\expectation{\a{\vpower}}$ and $\covariance{\a{\vpower}}$ are found using \equref{steady-power-mean} and \equref{steady-power-covariance}, respectively.
