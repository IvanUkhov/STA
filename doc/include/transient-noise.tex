Suppose the system at each moment of time is affected by a random noise, symbolically denoted by $\noise$. The nominal parameters of the platform and ambience are assumed to be known, however, due to the noise (caused by, for instance, varying environmental conditions), the actual dynamic power dissipation and, consequently, temperature fluctuations are uncertain.

Additional assumptions:
\begin{itemize}
  \item $\power_{ij}$ are deterministic $\forall i \in \nodeindex, j \in \timeindex$.
  \item $\noise$ is based on the white noise.
  \item The covariance matrix $\covariance{\noise}$ of the noise is known.
\end{itemize}

The nominal power dissipation is assumed to be fixed within each time interval $\timeinterval_i$, resulting in the constant term $\vpower$ instead of $\vpower(t)$ in \equref{fourier} for this time interval:
\begin{equation} \equlabel{fourier-constant-power}
  \mcapacitance \frac{d\vtemperature(\time)}{d\time} + \mconductance \vtemperature(\time) = \vpower
\end{equation}
The presence of the noise is modeled by an additional term, $\noise$, in the right-hand side of the equation above. Due to the assumptions, $\noise$ is a \mnrv\ with $\nodecount$ components (one for each thermal node), i.e., $\noise \sim \normal(\vzero, \covariance{\noise})$. Since the covariance matrix is a real symmetric matrix, it can be diagonalized using the eigenvalue decomposition as $\covariance{\noise} = \m{U} \m{\Lambda} \m{U}^T$, where $\m{U}$ is a orthogonal matrix of the eigenvectors and $\m{\Lambda}$ is a diagonal matrix of the eigenvalues \cite{press2007}. Consequently, the noise is decomposed into $\noise = \mnoisedeviation \vstdnormal_\time$, where $\mnoisedeviation = \m{U} \m{\Lambda}^{1/2}$ and the stochastic process $\{ \vstdnormal_\time \}_{\time \in \timeset}$ is the $\nodecount$-dimensional white noise \cite{oksendal2003}, a set of vectors of independent \snrvs, i.e., $\vstdnormal_\time \sim \normal(\vzero, \mone)$\footnote{$\mnoisedeviation \mone \mnoisedeviation^T = \covariance{\noise}$, therefore, $\mnoisedeviation \vstdnormal_\time = \m{U} \m{\Lambda}^{1/2} \vstdnormal_t \sim \normal(\vzero, \covariance{\noise})$.}. Hence, we have the following stochastic process:
\begin{equation} \equlabel{fourier-noise}
  \mcapacitance \frac{d\vtemperature_\time}{d\time} + \mconductance \vtemperature_\time = \vpower + \mnoisedeviation \vstdnormal_\time
\end{equation}
\equref{fourier-noise} can be rewritten in the differential form:
\begin{equation} \equlabel{fourier-wiener}
  d\vtemperature_\time = \mcapacitance^{-1}(\vpower - \mconductance \vtemperature_\time)d\time + \mcapacitance^{-1} \mnoisedeviation \vstdnormal_t d\time
\end{equation}
$\vstdnormal_\time d\time = d\atwo_\time$, where $\{ \atwo_\time \}_{\time \in \timeset}$ is known as the Wiener process or Brownian motion \cite{oksendal2003}.

The integration of \equref{fourier-wiener} cannot be done in the regular Riemann-Stieltjes sense, since the Brownian motion is nowhere differentiable. Therefore, the It\^{o} interpretation \cite{oksendal2003} of the stochastic differential equation in \equref{fourier-wiener} is used. It can be seen that the stochastic process $\{ \vtemperature_\time \}_{\time \in \timeset}$ resembles the Ornstein-Uhlenbeck process \cite{kloeden1992}. Thus, in order to solve \equref{fourier-wiener}, we apply the It\^{o} formula \cite{oksendal2003} to the function $\ecg{}{t} \vtemperature_\time$, where $e^\m{A}$ denotes the matrix exponential of $\m{A}$, and obtain:
\begin{align*}
  d(\ecgt \mtemperature_\time) & = \cg \ecgt \vtemperature_\time d\time + \ecgt d\vtemperature_\time \\
  & = \ecgt \mcapacitance^{-1} (\vpower d\time + \mnoisedeviation d\v{W}_t)
\end{align*}
The solution can be found by taking integrals on both sides:
\begin{align} \equlabel{solution-full}
  \vtemperature_\time = & \emcgt \vtemperature_0 - \\
    & (\cg)^{-1}(\emcgt - \mone) \mcapacitance^{-1} \vpower + \nonumber \\
    & \int_0^\time \ecg{}{(s - \time)} \mcapacitance^{-1} \mnoisedeviation \; d\atwo_s \nonumber
\end{align}
To simplify the further derivation, we introduce the following notation:
\begin{align}
  & \m{A}(\time) = \emcg{\time} \equlabel{a} \\
  & \m{B}(\time) = -(\cg)^{-1}(\emcg{\time} - \mone) \mcapacitance^{-1} \equlabel{b} \\
  & \v{D}_\time = \int_0^{\time} \ecg{}{(s - \time)} \mcapacitance^{-1} \mnoisedeviation \: d\atwo_s \equlabel{d}
\end{align}
Therefore, \equref{solution-full} can be rewritten:
\[
  \vtemperature_\time = \m{A}(\time) \vtemperature_0 + \m{B}(\time) \vpower + \v{D}_\time
\]
Since the Wiener process has independent normally distributed increments, an integration with respect to it, i.e., $\int_0^\time f(s) dW_s$, is a \nrv. It can also be shown that in this case the mean is zero and variance is equal to $\int_0^\time f^2(s) ds$. Hence, $\v{D}_\time \sim \normal(\vzero, \covariance{\v{D}_\time})$ is a \mnrv\ with a zero mean vector and the following covariance matrix:
\begin{align}
  \covariance{\v{D}_\time} & = \int_0^\time \ecg{}{(s - \time)} \mcapacitance^{-1} \mnoisedeviation \; ds \nonumber \\
    & = (\cg)^{-1} (\mone - \emcgt) \mcapacitance^{-1} \mnoisedeviation \equlabel{covariance-d}
\end{align}
Consequently, $\m{B}(\time) \vpower + \v{D}_\time \sim \normal(\m{B}(\time) \vpower, \: \covariance{\v{D}_\time})$.

The solution given by \equref{solution-full} is applicable for one time interval with constant $\vpower$. In order to model the whole period $\period$, the computations should be performed for each of the subintervals sequentially. Therefore, we have the following recurrent expression:
\begin{equation} \equlabel{fourier-recurrence-noise}
  \vtemperature_{i + 1} = \m{A}_i \vtemperature_i + \m{B}_i \vpower_i + \m{D}_i, \sep i \in \timeindex
\end{equation}
where we denote $\vtemperature_i = \vtemperature_{\time_i}$, $\vpower_i = \vpower_{\time_i}$, $\m{A}_i = \m{A}(\timeinterval_i)$, $\m{B}_i = \m{B}(\timeinterval_i)$, and $\m{D}_i = \m{D}_{\timeinterval_i}$. The initial temperature $\vtemperature_0$ is assumed to be known. The temperature vector $\vtemperature_i$ is the \mnrv\ obtained in the previous step of the iterative process.

Due to the properties of the Wiener process, $\vtemperature_i$ is independent of $\m{D}_i$. Therefore, $\vtemperature_{i+1}$ is a \mnrv:
\begin{align*}
  & \vtemperature_{i + 1} \sim \normal(\expectation{\vtemperature_{i + 1}}, \covariance{\vtemperature_{i + 1}}), \sep i \in \timeindex \\
  & \expectation{\vtemperature_{i+1}} = \m{A}_i \expectation{\vtemperature_i} + \m{B}_i \vpower_i \\
  & \covariance{\vtemperature_{i+1}} = \m{A}_i \covariance{\vtemperature_i} \m{A}^T_i + \covariance{\m{D}_i}
\end{align*}
\equref{a}, \equref{b}, and \equref{covariance-d} are used to perform the computations.

Alternatively, non-recurrent expressions can be obtained by performing an iterative repetition of \equref{fourier-recurrence-noise}:
\begin{equation*}
  \vtemperature_{i + 1} = \m{Q}_i \vtemperature_0 + \sum_{j = 0}^{i} \atwo_{ij} \m{B}_j \vpower_j + \sum_{j = 0}^{i} \atwo_{ij} \m{D}_j, \sep i \in \timeindex
\end{equation*}
where the following notation is introduced:
\begin{align*}
  & \aone_i = \m{A}_i \dots \m{A}_0 \\
  & \atwo_{ij} = \begin{cases}
    \m{A}_i \dots \m{A}_{j + 1} & \mbox{if } i \geq j + 1 \\
    \mone & \mbox{otherwise}
  \end{cases}
\end{align*}
Therefore:
\begin{align*}
  & \vtemperature_{i + 1} \sim \normal(\expectation{\vtemperature_{i + 1}}, \covariance{\vtemperature_{i + 1}}), \sep i \in \timeindex \\
  & \expectation{\vtemperature_{i + 1}} = \aone_i \vtemperature_0 + \sum_{j = 0}^{i} \atwo_{ij} \m{B}_j \vpower_j \\
  & \covariance{\vtemperature_{i + 1}} = \sum_{j = 0}^{i} \atwo_{ij} \covariance{\m{D}_j} \atwo_{ij}^T
\end{align*}
