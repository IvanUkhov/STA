Using \equref{tta-fourier-non-recurrent-noise}, the temperature vector at the end of the curves can be computed as:
\begin{align*}
  \vtemp_{\stepcount} & = \am{A}_{(\stepcount - 1)0} \vtemp_0 + \m{F}_{\stepcount - 1} + \v{L}_{\stepcount - 1}
\end{align*}
Note, as in \secref{tta-noise}, $\m{F}_i$ is computed according to \equref{f} with $\vmean{i} \equiv \vmean{\dyn \; i}$. Following the same procedure as in \secref{dssta-process-variation}, the vector of the initial temperature can be derived:
\[
  \vtemp_0 = \m{Q} \; (\m{F}_{\stepcount - 1} + \v{L}_{\stepcount - 1})
\]
where:
\begin{align}
  & \vtemp_0 \sim \normal(\expect{\vtemp_0}, \cov{\vtemp_0}) \equlabel{dssta-initial-temperature-noise} \\
  & \expect{\vtemp_0} = \m{Q} \; \v{F}_{\stepcount - 1} \nonumber \\
  & \cov{\vtemp_0} = \dbl{\m{Q}, \m{M}_{\stepcount - 1}} \nonumber
\end{align}
The rest of the temperature vectors are found using \equref{tta-fourier-recurrence-noise} and \equref{tta-probabilistic-recurrence-noise} with the initial temperature given above.

Alternatively, the non-recurrent expressions in \equref{tta-fourier-non-recurrent-noise} can be employed, where:
\[
  \cov{\vtemp_{i + 1}} = \dbl{\am{A}_{i0} \; \m{Q}, \m{M}_{\stepcount - 1}} + \m{M}_i
\]
and $\expect{\vtemp_0}$ is calculated according to \equref{dssta-initial-temperature-noise}.
