For the clarity reasons and without loss of generality, we let $\vtemp(\time) \equiv \vtemp(\time) - \vtemp_\amb$ in \equref{fourier}, since $\vtemp_\amb$ can be easily added to the final solution afterwards. In order to handle discrete power profiles and calculate the corresponding temperature profiles (see \secref{power-model}), we assume that within each time interval $\timeinterval_i$ the power consumption is constant in the right-hand side of \equref{fourier}. In this case, for a single time interal, we have a system of ordinary differential equations:
\begin{equation} \equlabel{fourier-constant-power}
  \mcapacitance \frac{d\vtemp(\time)}{d\time} + \mconductance \vtemp(\time) = \vpower
\end{equation}
The solution of \equref{fourier-constant-power} is the following:
\[
  \vtemp(\time) = \m{A}(\time) \vtemp(0) + \m{B}(\time) \vpower
\]
where:
\begin{align}
  & \m{A}(\time) = \emcg{\time} \equlabel{a} \\
  & \m{B}(\time) = -(\cg)^{-1}(\emcg{\time} - \mone) \mcapacitance^{-1} \equlabel{b}
\end{align}
Here $e^\m{M}$ denotes the matrix exponential of $\m{M}$. Consequently, the following recurrence can be applied in order to calculate temperature vectors for all $\timeinterval_i$ of the power profile:
\begin{equation} \equlabel{fourier-recurrence}
  \vtemp_{i + 1} = \m{A}_i \vtemp_i + \m{B}_i \vpower_i, \sep i \in \timeindex
\end{equation}
where we denote $\m{A}_i = \m{A}(\timeinterval_i)$ and $\m{B}_i = \m{B}(\timeinterval_i)$.
