\documentclass[conference]{IEEEtran}
\usepackage[english]{babel}
\usepackage[usenames,dvipsnames]{color}
\usepackage{amssymb}

\usepackage{multibib}

\newcommand{\equref}[1]{Eq.~(\ref{eq:#1})}
\newcommand{\secref}[1]{Sec.~\ref{sec:#1}}

\newcommand{\equlabel}[1]{\label{eq:#1}}
\newcommand{\seclabel}[1]{\label{sec:#1}}

\newcommand{\system}{\mathcal{S}}

\newcommand{\platform}{\mathcal{P}}
\newcommand{\coreset}{\platform}
\newcommand{\core}{\rho}
\newcommand{\coreindex}{I_\core}
\newcommand{\corecount}{N_\core}
\newcommand{\physics}{\mathcal{X}}

\newcommand{\application}{\mathcal{A}}
\newcommand{\edgeset}{\mathcal{E}}

\newcommand{\taskset}{\mathcal{V}}
\newcommand{\task}{v}
\newcommand{\taskindex}{I_\task}
\newcommand{\taskcount}{N_\task}

\newcommand{\mapping}{\phi}
\newcommand{\schedule}{\psi}
\newcommand{\real}{\mathbb{R}}
\newcommand{\positivereal}{\real^{+}}

\newcommand{\pdf}{f}

\newcommand{\timeset}{T}
\renewcommand{\time}{t}
\newcommand{\timeindex}{I_\time}
\newcommand{\timecount}{N_\time}

\newcommand{\outcomeset}{\Omega}
\newcommand{\outcome}{\omega}
\newcommand{\salgebra}{\mathcal{F}}
\newcommand{\pmeasure}{P}

\newcommand{\pspace}{(\outcomeset, \salgebra, \pmeasure)}

\newcommand{\m}[1]{\mathbf{#1}}
\renewcommand{\v}[1]{\mathbf{#1}}

\newcommand{\powerset}{U}
\newcommand{\power}{U}
\newcommand{\vpower}{\v{U}}

\newcommand{\temperatureset}{\Theta}
\newcommand{\temperature}{\Theta}
\newcommand{\vtemperature}{\v{\Theta}}

\newcommand{\mcapacitance}{\m{C}}
\newcommand{\mconductance}{\m{G}}
\newcommand{\mresistance}{\m{R}}

\newcommand{\nodecount}{N_n}

\newcommand{\apply}[1]{, \hspace{10pt} #1}

\newcommand{\period}{\Pi}
\newcommand{\executiontime}{\pi}

\newcommand{\taskpower}{U}
\newcommand{\tasktime}{T}


\begin{document}
  \title{Stochastic Temperature Analysis}
  \author{Ivan Ukhov}

  \maketitle

  \begin{abstract}
    Temperature is important.
  \end{abstract}

  \section{Introduction} \seclabel{introduction} \seclabel{uncertain-parameters}
  A multiprocessor system is a subject of uncertainties. The major sources of uncertainties are:
\begin{itemize}
  \item Process variation of the platform.
  \item Various input data of the application.
\end{itemize}
Therefore, the dynamic power profile and, consequently, temperature profile are uncertain.


  \section{Preliminaries}
  \subsection{Architecture Model}
Consider a \definition{multiprocessor system} $\system = (\platform, \physics)$. $\platform = \{ \core_i \}_{i \in \coreindex}$ is a joint set of programmable processors and communication buses with a single index set $\coreindex = \{0, \dots, \corecount - 1\}$. Without loss of generality, we do not treat processors and buses separately and shall refer to them as \definition{cores}. $\physics$ denotes a set of \definition{physical characteristics} of the platform and ambience, which includes the floorplan of the cores, geometry of the die and thermal package, thermal parameters of the ambience and materials that the die and package are made of.

\subsection{Power and Thermal Models}
Given $\system$, an \definition{equivalent thermal RC circuit} of the system is constructed \cite{kreith2000}. The circuit is composed of a set of \definition{thermal nodes} denoted by $\nodeset = \{ \node_i \}_{i \in \nodeindex}$, where $\nodeindex = \{ 0, \dots, \nodecount - 1 \}$ is an index set. The structure of the circuit depends on the desired level of details, i.e., accuracy. If a node belongs to a core, it is called \definition{active}, otherwise, \definition{passive}.

The dynamic power dissipation of the constructed circuit at time $\time$ is denoted by a column vector $\vpower(\time) = \vector{\power_i(\time)}_{i \in \nodeindex}$ (hereafter this notation is used to emphasize elements of a vector), where $\power_i(\time)$ is the power of the $i$th node. Passive nodes dissipate no power. The corresponding temperature variation is denoted by $\vtemperature(\time) = \vector{\temperature_i(\time)}_{i \in \nodeindex}$, where $\temperature_i(\time)$ is the temperature of the $i$th thermal node.

The thermal behaviour of the system is modeled with the following heat equation:
\begin{equation} \equlabel{fourier}
  \mcapacitance \frac{d\vtemperature(\time)}{d\time} + \mconductance (\vtemperature(\time) - \vtemperature_{amb}) = \vpower(\time)
\end{equation}
where $\mcapacitance$ and $\mconductance$ are $\nodecount \times \nodecount$ matrices of the thermal capacitance and conductance, respectively. $\vtemperature_{amb}$ is the ambient temperature. Without loss of generality, let $\vtemperature_{amb} = 0$.

A discrete \definition{dynamic power profile} of the system over a time interval $\period$ is defined as a tuple $\powerprofile = (\timepartition, \mpower)$. $\timepartition = \{ \timeinterval_i \}_{i \in \timeindex}$ is a partition of $\period$ into $\stepcount$ subintervals, indexed by $\timeindex = \{ 0, \dots, \stepcount - 1 \}$, such that $\sum_{i \in \timeindex} \timeinterval_i = \period$. $\mpower = \matrix{\power_{ij}}_{i \in \nodeindex, j \in \timeindex}$ is a $\nodecount \times \stepcount$ matrix of the dynamic power dissipation of each thermal node $\node_i$ in each time interval $\timeinterval_j$. $\vpower_j = \vector{\power_{ij}}_{i \in \nodeindex}$ denotes the $j$th column of $\mpower$, i.e., the vector of the dynamic power dissipation of all nodes in the $j$th time interval.

A discrete \definition{temperature profile} $\temperatureprofile$ with respect to a power profile $\powerprofile$ is defined as a tuple $\temperatureprofile = (\timepartition, \mtemperature)$, where the time partition $\timepartition$ remains the same as for $\powerprofile$ and $\mtemperature = \matrix{\vtemperature_{ij}}_{i \in \nodeindex, j \in \timeindex}$ is a matrix of the corresponding temperature values for each thermal node $\node_i$ in each time interval $\timeinterval_j$. $\vtemperature_j = \vector{\temperature_{ij}}_{i \in \nodeindex}$ denotes the $j$th column of $\mtemperature$, i.e., the vector of temperature values of all nodes in the $j$th time interval.

\subsection{Stochastic Process}
Let $\pspace$ be a \definition{probability space} \cite{oksendal2003}. $\outcomeset$ is a set of outcomes, $\salgebra$ is a $\sigma$-algebra on $\outcomeset$, and $\pmeasure$ is a probability measure. A $\salgebra$-feasible function $X: \outcome \to \real$ is called a \definition{\rv}. $X \sim \normal(\mean{X}, \deviation{X}^2)$ denotes a \definition{normal distribution}, where $\mean{X}$ and $\deviation{X}^2$ are the \definition{mean} and \definition{variance}, respectively. A vector of \rvs\ $\v{X}$ is said to have a \definition{multivariate normal distribution} if any linear combination of its component is a (univariate) normal variable, this fact is denoted as $\v{X} \sim \normal(\expectation{\v{X}}, \covariance{\v{X}})$, where $\expectation{\v{X}}$ and $\covariance{\v{X}}$ are the \definition{mean vector} and \definition{covariance matrix}, respectively.

A \definition{stochastic process} is a parametrized collection of \rvs\ $\{ X_\time \}_{\time \in \timeset}$, where $\timeset$ is the parameter space, the half line $[0, \infty)$ meaning time. $\{ \v{X}_\time \}_{\time \in \timeset}$ denotes a \definition{multidimensional stochastic process}.

\subsection{Process Variation} \seclabel{process-variation}
Due to the process variation, the actual power consumption of the system can deviate from the nominal values. Therefore, the dynamic power profile of the system is assumed to be uncertain, i.e., the power dissipation $\power_{ij}$ of the $i$th core in the $j$th time interval is a \rv.

Assume that the uncertainties, introduced by the process variation into a chip, do not depend on time, i.e., a core within a chip, performing a curtain operation, always exhibits the same deviation from the nominal value. Also, due to the nature of the process variation, the direction of the deviation of a core is fixed, i.e., a core can always be either \definition{cold}, meaning that its power always tends to be lower than the nominal value, or \definition{hot}, meaning that its power always tends to be higher than the nominal value.


  \section{Static Steady-State Temperature Analysis} \seclabel{sssta}
  Suppose that we are given a periodic dynamic power profile and are interested in the thermal behaviour of the system in the long run. In this case, it is assumed that, after a sufficiently long period of time, the system reaches such a state, where temperature either preserves a constant value or exhibits a periodic pattern. Concequently, the state is called either the \definition{static steady state} or \definition{dynamic steady state}, respectiely. Therefore, there are two types of the steady-state temperature analysis: \definition{static steady-state temperature analysis} (SSSTA), discussed in this section, and \definition{dynamic steady-state temperature analysis} (DSSTA), discussed in \secref{dssta}.

The SSSTA delivers a single vector of temperature values for each of the thermal nodes of the system.

\subsection{Problem Formulation}
The source of uncertainties is the process variation.

Given:
\begin{itemize}
  \item A multiprocessor system $\system$.
  \item A nominal dynamic power profile $\powerprofile_\dyn$.
  \item The variance $\deviation{ij}^2$ of $\power_{ij} \; \forall i \in \nodeindex, j \in \timeindex$.
\end{itemize}

Find:
\begin{itemize}
  \item The probability distribution of the temperature profile $\temperatureprofile$ with respect to $\powerprofile$, when the static steady state is reached.
\end{itemize}

\subsection{Solution} \seclabel{sss-solution}
Additional assumptions:
\begin{itemize}
  \item $\power_{ij} \sim \normal(\mean{ij}, \deviation{ij}^2)$, i.e., the power dissipation is distributed normally.
  \item $\power_{i_1 j_1}$ and $\power_{i_2 j_2}$ are independent $\forall i_1, i_2 \in \nodeindex,  i_1 \neq i_2$ and $\forall j_1, j_2 \in \timeindex$.
\end{itemize}

As discussed earlier, the system is supposed to reach a hypothetical thermal balance, where temperature does not change over the time. In this case, the power profile is averaged and given as a constant input to the thermal RC circuit. Therefore, \equref{fourier} does not depend on time and drops the derivative term:
\begin{equation} \equlabel{sss-fourier}
  \a{\vtemperature} = \mconductance^{-1} \a{\vpower}
\end{equation}
where $\a{\vpower}$ is a random vector that represents the average power dissipation of thermal nodes, consequently, $\a{\vtemperature}$ is a random vector that corresponds to the average temperature. As mentioned in \secref{thermal-model}, the ambient temperature, $\vtemperature_\amb$, should be added to the right-hand side of \equref{sss-fourier} in order to obtain the actual temperature values.

$\a{\vpower}$ is computed with an ordinal equation:
\[
  \a{\vpower} = \frac{1}{\stepcount} \sum_{j \in \timeindex} \vpower_j
\]
where $\vpower_j$ are \mnrvs. As mentioned in \secref{uncertainties}, \rvs\ $\power_{ij}$ for a fixed $i$ and $\forall j$ (a row in the power matrix $\mpower$) are required to jointly deviate from the nominal (mean) values into the same direction. Therefore:
\[
  \a{\vpower} = \frac{1}{\stepcount} \sum_{j \in \timeindex} \vmean{j} + \frac{1}{\stepcount} \sum_{j \in \timeindex} \ddeviation{j} \; \vstdnormal
\]
It can be seen that $\a{\vpower}$ is a \mnrv\ with independent components:
\begin{align}
  & \a{\vpower} \sim \normal(\expectation{\a{\vpower}}, \covariance{\a{\vpower}}) \nonumber \\
  & \expectation{\a{\vpower}} = \frac{1}{\stepcount} \sum_{j \in \timeindex} \vmean{j} \equlabel{steady-power-mean} \\
  & \covariance{\a{\vpower}} = \frac{1}{\stepcount^2} \left[ \sum_{j \in \timeindex} \ddeviation{j} \right]^2 \equlabel{steady-power-covariance}
\end{align}
Consequently, after the linear transformation of $\a{\mpower}$ in \equref{sss-fourier}, $\a{\mtemperature}$ also becomes a \mnrv:
\begin{align*}
  & \a{\vtemperature} \sim \normal(\expectation{\a{\vtemperature}}, \covariance{\a{\vtemperature}}) \\
  & \expectation{\a{\vtemperature}} = \mconductance^{-1} \expectation{\a{\vpower}} \\
  & \covariance{\a{\vtemperature}} = \mconductance^{-1} \covariance{\a{\vpower}} (\mconductance^{-1})^T
\end{align*}
where $\expectation{\a{\vpower}}$ and $\covariance{\a{\vpower}}$ are found using \equref{steady-power-mean} and \equref{steady-power-covariance}, respectively.


  \section{Transient Temperature Analysis} \seclabel{tta}
  \subsection{Problem Formulation}
Consider a multiprocessor system affected by a random noise. The nominal parameters of the platform, application, and ambience are assumed to be known. However, due to the noise (caused by, for instance, the process variation, various ambience conditions, etc.) the actual dynamic power dissipation and, consequently, temperature fluctuations are uncertain. The noise is represented by a multivariate normal random variable, symbolically denoted as $\noise$, with $\nodecount$ components corresponding to each of the thermal node, i.e., $\noise \sim \normal(\vzero, \covariance{\noise})$.

Given:
\begin{itemize}
  \item A multiprocessor platform $\platform$.
  \item Nominal physical parameters $\physics$.
  \item A nominal dynamic power profile $\powerprofile$.
  \item The covariance matrix of the noise $\covariance{\noise}$.
\end{itemize}

Find:
\begin{itemize}
  \item The probability distribution of the (stochastic) temperature profile $\temperatureprofile$ of the platform with respect to $\powerprofile$.
\end{itemize}

\subsection{Model}
Since the input to the analysis is a discrete power profile, the nominal power is assumed to be fixed within each of the time intervals $\timepartition$. Therefore, when modeling one time interval, $\vpower(t)$ is constant in \equref{fourier}, i.e., $\vpower(t) = \vpower$:
\[
  \mcapacitance \frac{d\vtemperature(\time)}{d\time} + \mconductance \vtemperature(\time) = \vpower
\]
In order to model the presence of noise, we introduce an additional term, $\noise$, in the right-hand side of the equation above. $\noise$ is decomposed into $\mnoisedeviation \vnoise_t$, where $\vnoise_t$ is a vector of independent standard normal random variables, i.e., $\vnoise_\time \sim \normal(\vzero, \mone)$, and  $\mnoisedeviation = \m{U} \m{\Lambda}^{1/2}$. The matrices $\m{U}$ and $\m{\Lambda}$ are found using the eigenvalue decomposition of the covariance matrix, i.e., $\covariance{\noise} = \m{U} \m{\Lambda} \m{U}^T$. Note that $\mnoisedeviation \mone \mnoisedeviation^T = \covariance{\noise}$, therefore, $\m{U} \m{\Lambda}^{1/2} \vnoise_t \sim \normal(\vzero, \covariance{\noise})$. Hence, denoting time-dependent (stochastic) processes with the subscript $\time$, we have:
\begin{equation} \equlabel{fourier-noise}
  \mcapacitance \frac{d\vtemperature_\time}{d\time} + \mconductance \vtemperature_\time = \vpower + \mnoisedeviation \vnoise_t
\end{equation}
The stochastic process $\{ \vnoise_\time \}_{\time \in \timeset}$ is known as the white noise \cite{oksendal2003}, in this case, multidimensional. \equref{fourier-noise} can be rewritten in the differential form:
\begin{equation} \equlabel{fourier-wiener}
  d\vtemperature_\time = \mcapacitance^{-1}(\vpower - \mconductance \vtemperature_\time)d\time + \mcapacitance^{-1} \mnoisedeviation \vnoise_t d\time
\end{equation}
Note that $\vnoise_\time d\time = d\m{W}_\time$ where $\{ \m{W}_\time \}_{\time \in \timeset}$ is the Wiener process \cite{oksendal2003}. Also it can be seen that the stochastic process $\{ \vtemperature_\time \}$ resembles the Ornstein-Uhlenbeck process \cite{kloeden1992}. The It\^{o} interpretation \cite{oksendal2003} of the last equations is a stochastic process $\{ \vtemperature_\time \}_{\time \in \timeset}$ that satisfies the following integral equation:
\[
  \vtemperature_\time = \vtemperature_0 + \int_0^\time a(s, \vtemperature_s)ds + \int_0^\time b(s, \vtemperature_s) d\m{W}_\time
\]
In order to solve \equref{fourier-wiener} and find $\vtemperature_\time$, we apply the It\^{o} formula to the function $\ecg{}{t} \vtemperature_\time$, where $e^\m{A}$ is the matrix exponential of $\m{A}$, and obtain:
\begin{align*}
  d(\ecgt \mtemperature_\time) & = \cg \ecgt \vtemperature_\time d\time + \ecgt d\vtemperature_\time \\
  & = \ecgt \mcapacitance^{-1} (\vpower d\time + \mnoisedeviation d\v{W}_t)
\end{align*}
The solution can be found by taking integrals on both sides:
\begin{align} \equlabel{solution-full}
  \vtemperature_\time = & \emcgt \vtemperature_0 - \\
    & (\cg)^{-1}(\emcgt - \mone) \mcapacitance^{-1} \vpower + \nonumber \\
    & \int_0^\time \ecg{}{(s - \time)} \mcapacitance^{-1} \mnoisedeviation \; d\m{W}_s \nonumber
\end{align}
To simplify the further derivation, we introduce the following notation:
\begin{align}
  & \m{A}(\time) = \emcg{\time} \equlabel{a} \\
  & \m{B}(\time) = -(\cg)^{-1}(\emcg{\time} - \mone) \mcapacitance^{-1} \equlabel{b} \\
  & \v{D}_\time = \int_0^{\time} \ecg{}{(s - \time)} \mcapacitance^{-1} \mnoisedeviation \: d\m{W}_s \equlabel{d}
\end{align}
Therefore, \equref{solution-full} can be rewritten:
\[
  \vtemperature_\time = \m{A}(\time) \vtemperature_0 + \m{B}(\time) \vpower + \v{D}_\time
\]
Since the Wiener process has independent normally distributed increments, an integration with respect to it, i.e., $\int_0^\time f(s) dW_s$, is a normal random variable. It can also be shown that in this case the mean is zero and variance is equal to $\int_0^\time f^2(s) ds$. Hence, $\v{D}_\time \sim \normal(\vzero, \covariance{\v{D}_\time})$ is a multivariate normal random variable with a zero mean vector and the following covariance matrix:
\begin{align}
  \covariance{\v{D}_\time} & = \int_0^\time \ecg{}{(s - \time)} \mcapacitance^{-1} \mnoisedeviation \; ds \nonumber \\
    & = (\cg)^{-1} (\mone - \emcgt) \mcapacitance^{-1} \mnoisedeviation \equlabel{covariance-d}
\end{align}
The components of $\v{D}_\time$ in general are not independent. Consequently, $\m{B}(\time) \vpower + \v{D}_\time \sim \normal(\m{B}(\time) \vpower, \: \covariance{\v{D}_\time})$.

The solution given by \equref{solution-full} is applicable for one time interval with constant $\vpower$. In order to model the whole period $\totaltimeinterval$, the computations should be performed for each of the subintervals sequentially. Therefore, we have the following recurrent expression:
\begin{equation} \equlabel{recurrence}
  \vtemperature_{\time_{i + 1}} = \m{A}(\timeinterval_i) \vtemperature_{\time_i} + \m{B}(\timeinterval_i) \vpower_{\time_i} + \m{D}_{\timeinterval_i}
\end{equation}
Here $\vtemperature_{\time_i}$ is the multivariate normal random variable obtained in the previous step of the iterative process. Thus, $\m{A}(\timeinterval_i) \vtemperature_{\time_i}$ is a multivariate normal variable as well with:
\begin{align*}
  & \mean{\m{A}(\timeinterval_i) \vtemperature_{\time_i}} = \m{A}(\timeinterval_i)\mean{\vtemperature_{\time_i}} \\
  & \covariance{\m{A}(\timeinterval_i) \vtemperature_{\time_i}} = \m{A}(\timeinterval_i) \covariance{\vtemperature_{\time_i}} \m{A}^T(\timeinterval_i)
\end{align*}
Due to the properties of the Wiener process, $\vtemperature_{\time_i}$ is independent of $\m{D}_{\timeinterval_i}$. Therefore, $\vtemperature_{\time_{i+1}}$ is a multivariate normal random variable with the following mean and covariance:
\begin{align*}
  & \mean{\vtemperature_{\time_{i+1}}} = \m{A}(\timeinterval_i)\mean{\vtemperature_{\time_i}} + \m{B}(\timeinterval_i) \vpower_{\time_i} \\
  & \covariance{\vtemperature_{\time_{i+1}}} = \m{A}(\timeinterval_i) \covariance{\vtemperature_{\time_i}} \m{A}^T(\timeinterval_i) + \covariance{\m{D}_{\timeinterval_i}}
\end{align*}
\equref{a}, \equref{b}, and \equref{covariance-d} can be used to perform the computations.


  \section{Computation Improvements}
  \subsection{Power Profile Partitioning} \seclabel{power-partitioning}
  As mentioned earlier in \secref{tta-task-model}, $\timepartition$ can be defined arbitrary, depending on the desired level of accuracy.

  \subsubsection{Even Partitioning}
  $\period$ can be split evenly, so that the coefficients $\m{A}$ (\equref{a}) and $\m{B}$ (\equref{b}) are constant, i.e., computed only once.

  \subsubsection{Event-Driven Partitioning}
  $\period$ can be split only in those moments where there is a change in the task placement, i.e., in the beginnings and ends of tasks. If the number of task is modest and their durations are sufficiently long, it may end up having only a few time intervals to compute.

  \bibliographystyle{unsrt}
  \bibliography{include/references}
\end{document}
