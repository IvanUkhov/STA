\documentclass[conference]{IEEEtran}
\usepackage[english]{babel}
\usepackage[usenames,dvipsnames]{color}
\usepackage{amssymb}

\usepackage{multibib}

\newcommand{\equref}[1]{Eq.~(\ref{eq:#1})}
\newcommand{\secref}[1]{Sec.~\ref{sec:#1}}

\newcommand{\equlabel}[1]{\label{eq:#1}}
\newcommand{\seclabel}[1]{\label{sec:#1}}

\newcommand{\system}{\mathcal{S}}

\newcommand{\platform}{\mathcal{P}}
\newcommand{\coreset}{\platform}
\newcommand{\core}{\rho}
\newcommand{\coreindex}{I_\core}
\newcommand{\corecount}{N_\core}
\newcommand{\physics}{\mathcal{X}}

\newcommand{\application}{\mathcal{A}}
\newcommand{\edgeset}{\mathcal{E}}

\newcommand{\taskset}{\mathcal{V}}
\newcommand{\task}{v}
\newcommand{\taskindex}{I_\task}
\newcommand{\taskcount}{N_\task}

\newcommand{\mapping}{\phi}
\newcommand{\schedule}{\psi}
\newcommand{\real}{\mathbb{R}}
\newcommand{\positivereal}{\real^{+}}

\newcommand{\pdf}{f}

\newcommand{\timeset}{T}
\renewcommand{\time}{t}
\newcommand{\timeindex}{I_\time}
\newcommand{\timecount}{N_\time}

\newcommand{\outcomeset}{\Omega}
\newcommand{\outcome}{\omega}
\newcommand{\salgebra}{\mathcal{F}}
\newcommand{\pmeasure}{P}

\newcommand{\pspace}{(\outcomeset, \salgebra, \pmeasure)}

\newcommand{\m}[1]{\mathbf{#1}}
\renewcommand{\v}[1]{\mathbf{#1}}

\newcommand{\powerset}{U}
\newcommand{\power}{U}
\newcommand{\vpower}{\v{U}}

\newcommand{\temperatureset}{\Theta}
\newcommand{\temperature}{\Theta}
\newcommand{\vtemperature}{\v{\Theta}}

\newcommand{\mcapacitance}{\m{C}}
\newcommand{\mconductance}{\m{G}}
\newcommand{\mresistance}{\m{R}}

\newcommand{\nodecount}{N_n}

\newcommand{\apply}[1]{, \hspace{10pt} #1}

\newcommand{\period}{\Pi}
\newcommand{\executiontime}{\pi}

\newcommand{\taskpower}{U}
\newcommand{\tasktime}{T}


\begin{document}
  \title{System-Level Stochastic Temperature Analysis of Multiprocessor Systems}
  \author{Ivan Ukhov, Petru Eles, and Zebo Peng}

  \maketitle

  \begin{abstract}
    In this work, we develop a novel technique to perform the temperature analysis of multiprocessor systems that are exposed to multiple sources of uncertainties. The uncertainties are manifested in the deviation of the actual power dissipation of the system from the nominal value. Apart from traditional time-invariant variations due to the process variation, we model the environmental noise, which by definition evolves in time. In order to demonstrate our technique in action, we perform an energy optimization under probabilistic temperature constraints. The experimental results show that the energy efficiency of the system can be significantly improved by accepting a negligibly small probability of the constraint violation.
  \end{abstract}

  \section{Introduction} \seclabel{introduction} \seclabel{uncertain-parameters}
  A multiprocessor system is a subject of uncertainties. The major sources of uncertainties are:
\begin{itemize}
  \item Process variation of the platform.
  \item Various input data of the application.
\end{itemize}
Therefore, the dynamic power profile and, consequently, temperature profile are uncertain.


  \section{Preliminaries}
  \subsection{Architecture Model}
Consider a \definition{multiprocessor system} $\system = (\platform, \physics)$. $\platform = \{ \core_i \}_{i \in \coreindex}$ is a joint set of programmable processors and communication buses with a single index set $\coreindex = \{0, \dots, \corecount - 1\}$. Without loss of generality, we do not treat processors and buses separately and shall refer to them as \definition{cores}. $\physics$ denotes a set of \definition{physical characteristics} of the platform and ambience, which includes the floorplan of the cores, geometry of the die and thermal package, thermal parameters of the ambience and materials that the die and package are made of.

\subsection{Power and Thermal Models}
Given $\system$, an \definition{equivalent thermal RC circuit} of the system is constructed \cite{kreith2000}. The circuit is composed of a set of \definition{thermal nodes} denoted by $\nodeset = \{ \node_i \}_{i \in \nodeindex}$, where $\nodeindex = \{ 0, \dots, \nodecount - 1 \}$ is an index set. The structure of the circuit depends on the desired level of details, i.e., accuracy. If a node belongs to a core, it is called \definition{active}, otherwise, \definition{passive}.

The dynamic power dissipation of the constructed circuit at time $\time$ is denoted by a column vector $\vpower(\time) = \vector{\power_i(\time)}_{i \in \nodeindex}$ (hereafter this notation is used to emphasize elements of a vector), where $\power_i(\time)$ is the power of the $i$th node. Passive nodes dissipate no power. The corresponding temperature variation is denoted by $\vtemperature(\time) = \vector{\temperature_i(\time)}_{i \in \nodeindex}$, where $\temperature_i(\time)$ is the temperature of the $i$th thermal node.

The thermal behaviour of the system is modeled with the following heat equation:
\begin{equation} \equlabel{fourier}
  \mcapacitance \frac{d\vtemperature(\time)}{d\time} + \mconductance (\vtemperature(\time) - \vtemperature_{amb}) = \vpower(\time)
\end{equation}
where $\mcapacitance$ and $\mconductance$ are $\nodecount \times \nodecount$ matrices of the thermal capacitance and conductance, respectively. $\vtemperature_{amb}$ is the ambient temperature. Without loss of generality, let $\vtemperature_{amb} = 0$.

A discrete \definition{dynamic power profile} of the system over a time interval $\period$ is defined as a tuple $\powerprofile = (\timepartition, \mpower)$. $\timepartition = \{ \timeinterval_i \}_{i \in \timeindex}$ is a partition of $\period$ into $\stepcount$ subintervals, indexed by $\timeindex = \{ 0, \dots, \stepcount - 1 \}$, such that $\sum_{i \in \timeindex} \timeinterval_i = \period$. $\mpower = \matrix{\power_{ij}}_{i \in \nodeindex, j \in \timeindex}$ is a $\nodecount \times \stepcount$ matrix of the dynamic power dissipation of each thermal node $\node_i$ in each time interval $\timeinterval_j$. $\vpower_j = \vector{\power_{ij}}_{i \in \nodeindex}$ denotes the $j$th column of $\mpower$, i.e., the vector of the dynamic power dissipation of all nodes in the $j$th time interval.

A discrete \definition{temperature profile} $\temperatureprofile$ with respect to a power profile $\powerprofile$ is defined as a tuple $\temperatureprofile = (\timepartition, \mtemperature)$, where the time partition $\timepartition$ remains the same as for $\powerprofile$ and $\mtemperature = \matrix{\vtemperature_{ij}}_{i \in \nodeindex, j \in \timeindex}$ is a matrix of the corresponding temperature values for each thermal node $\node_i$ in each time interval $\timeinterval_j$. $\vtemperature_j = \vector{\temperature_{ij}}_{i \in \nodeindex}$ denotes the $j$th column of $\mtemperature$, i.e., the vector of temperature values of all nodes in the $j$th time interval.

\subsection{Stochastic Process}
Let $\pspace$ be a \definition{probability space} \cite{oksendal2003}. $\outcomeset$ is a set of outcomes, $\salgebra$ is a $\sigma$-algebra on $\outcomeset$, and $\pmeasure$ is a probability measure. A $\salgebra$-feasible function $X: \outcome \to \real$ is called a \definition{\rv}. $X \sim \normal(\mean{X}, \deviation{X}^2)$ denotes a \definition{normal distribution}, where $\mean{X}$ and $\deviation{X}^2$ are the \definition{mean} and \definition{variance}, respectively. A vector of \rvs\ $\v{X}$ is said to have a \definition{multivariate normal distribution} if any linear combination of its component is a (univariate) normal variable, this fact is denoted as $\v{X} \sim \normal(\expectation{\v{X}}, \covariance{\v{X}})$, where $\expectation{\v{X}}$ and $\covariance{\v{X}}$ are the \definition{mean vector} and \definition{covariance matrix}, respectively.

A \definition{stochastic process} is a parametrized collection of \rvs\ $\{ X_\time \}_{\time \in \timeset}$, where $\timeset$ is the parameter space, the half line $[0, \infty)$ meaning time. $\{ \v{X}_\time \}_{\time \in \timeset}$ denotes a \definition{multidimensional stochastic process}.

\subsection{Process Variation} \seclabel{process-variation}
Due to the process variation, the actual power consumption of the system can deviate from the nominal values. Therefore, the dynamic power profile of the system is assumed to be uncertain, i.e., the power dissipation $\power_{ij}$ of the $i$th core in the $j$th time interval is a \rv.

Assume that the uncertainties, introduced by the process variation into a chip, do not depend on time, i.e., a core within a chip, performing a curtain operation, always exhibits the same deviation from the nominal value. Also, due to the nature of the process variation, the direction of the deviation of a core is fixed, i.e., a core can always be either \definition{cold}, meaning that its power always tends to be lower than the nominal value, or \definition{hot}, meaning that its power always tends to be higher than the nominal value.


  \section{Model Reduction} \seclabel{model-reduction}
  The number of thermal nodes in the equivalent RC thermal circuit can be extremely large. The first cause is the constantly increasing number of processing elements on a single die, each of which require a separate thermal node. The second reason is the desire of having more accurate models where a whole set of thermal nodes is to be mapped onto a single core. The third reason is the need to take into consideration the thermal package of the system, which usually has several layers with the same level of details as the die itself dramatically multiplying the number of thermal nodes.

A detailed thermal RC circuit produces a detailed temperature profile. However, in practice, what really matters is a relatively small set of thermal nodes, i.e., $\vOutTemp$ has a much smaller dimension than $\vTemp$ ($\onodes \ll \nodes$) in \equref{fourier}. For instance, one can be interested only in the temperature of the processing elements leaving the thermal package aside; in this case $\mOut = \mIn$. At the same time, the thermal package cannot be simply excluded from the model since it is essential for the dynamics of the system.

Moreover, as the first step, a complex model is required to be constructed to aim at a high accuracy, but it does not mean that a smaller set of state variables cannot achieve the same accuracy; it is just not that intuitive to be obtained.

All in all, there is a large room for model reduction techniques \cite{antoulas2001} where the basic idea is to project the state space of the initial system to a space of a lower dimension while preserving the original behaviour. Consequently, \equref{fourier} is transformed into:
\begin{align}
  & \r{\mCap} \frac{d\r{\vTemp}(\time)}{d\time} + \r{\mCond} \r{\vTemp}(\time) = \r{\mIn} \vPower(\time)  \equlabel{reduced-fourier} \\
  & \vOutTemp(\time) = \r{\mOut}^T \r{\vTemp}(\time) \nonumber
\end{align}
Here $\r{\vTemp} \in \real^{\rnodes}$, $\r{\mCap} \in \real^{\rnodes \times \rnodes}$, $\r{\mCond} \in \real^{\rnodes \times \rnodes}$, $\r{\mIn} \in \real^{\rnodes \times \inodes}$, and $\r{\mOut} \in \real^{\rnodes \times \onodes}$ where $\rnodes < \nodes$.


  \section{Solution to the Thermal Model} \seclabel{thermal-model-solution}
  For the clarity reasons and without loss of generality, we let $\vtemp(\time) \equiv \vtemp(\time) - \vtemp_\amb$ in \equref{fourier}, since $\vtemp_\amb$ can be easily added to the final solution afterwards. In order to handle discrete power profiles and calculate the corresponding temperature profiles (see \secref{power-model}), we assume that within each time interval $\timeinterval_i$ the power consumption is constant in the right-hand side of \equref{fourier}. In this case, for a single time interal, we have a system of ordinary differential equations:
\begin{equation} \equlabel{fourier-constant-power}
  \mcapacitance \frac{d\vtemp(\time)}{d\time} + \mconductance \vtemp(\time) = \vpower
\end{equation}
The solution of \equref{fourier-constant-power} is the following:
\[
  \vtemp(\time) = \m{A}(\time) \vtemp(0) + \m{B}(\time) \vpower
\]
where:
\begin{align}
  & \m{A}(\time) = \emcg{\time} \equlabel{a} \\
  & \m{B}(\time) = -(\cg)^{-1}(\emcg{\time} - \mone) \mcapacitance^{-1} \equlabel{b}
\end{align}
Here $e^\m{M}$ denotes the matrix exponential of $\m{M}$. Consequently, the following recurrence can be applied in order to calculate temperature vectors for all $\timeinterval_i$ of the power profile:
\begin{equation} \equlabel{fourier-recurrence}
  \vtemp_{i + 1} = \m{A}_i \vtemp_i + \m{B}_i \vpower_i, \sep i \in \timeindex
\end{equation}
where we denote $\m{A}_i = \m{A}(\timeinterval_i)$ and $\m{B}_i = \m{B}(\timeinterval_i)$.


  \section{Model of the Process Variation} \seclabel{process-variation}
  \subsection{Dynamic Power}
We assume that the variations in the dynamic power are normally distributed and are given in terms of the ratio between the standard deviation and expected value, i.e., for the $i$th thermal node we are given a constant $\ratio_{\dyn \; i}$ such that $\deviation{}/\mean{} = \ratio_{\dyn \; i}$ for any nominal value $\mean{}$ of the dynamic power. Let $\vratio_\dyn = \vector{\ratio_{\dyn \; i}}_{i \in \nodeindex}$ be a vector of the variation ratio for all thermal nodes.

Apart from the quantitative measure of variations $\vratio_\dyn$, we assume that the dependencies between different thermal nodes are also given and expressed in terms of their correlation matrix, denoted by $\mcorrelation{\dyn} = \matrix{\correlation{ij}}_{i,j \in \nodeindex}$. A correlation matrix is a normalized version of the corresponding covariance matrix, where $(i,j)$th element of the later is divided by $\deviation{X_i} \deviation{X_j}$, loosing the units of measure, but preserving the degree of linear dependencies in terms of the Person correlation coefficient\footnote{The Pearson correlation coefficient takes values from $-1$ to $1$ and is defined by $\rho_{ij} = \expectation{(X_i - \mean{X_i})(X_j - \mean{X_j})}/(\deviation{X_i}\deviation{X_j})$. If this information is not available, which is usually the case on early design stages, $\correlation{\dyn}$ can be assumed to be is equal to the identity matrix $\mone$.}.

Taking everything together, the dynamic power is modeled as the following:
\begin{align} \equlabel{dynamic-power}
  \vpower_\dyn(\time) & = \vmean{\dyn}(\time) + \diag{\vratio_\dyn} \diag{\vmean{\dyn}(\time)} \vnormal_\dyn \\
  & = \vmean{\dyn}(\time) + \ddeviation{\dyn}(\time) \; \vnormal_\dyn \nonumber
\end{align}
where $\vmean{\dyn}(\time)$ is the vector of the nominal (mean) dynamic power dissipation, $\diag{\v{X}}$ denotes a diagonal matrix with the central diagonal composed of the elements of the vector $\v{X}$, and $\vnormal_\dyn \sim \normal(\vzero, \mcorrelation{\dyn})$. Such a model with a single \mrv\ implies that the direction of the variations from the nominal values is fixed, i.e., a core always is either cold, meaning that its power always tends to be lower than the nominal value, or hot, meaning that its power always tends to be higher than the nominal value.

\subsection{Leakage Power}
In this paper, we assume that the leakage currents have a pure random nature and, hence, have no spacial correlations. Since the number of devices, which leak power, is considerably large in a modern processing unit, the total leakage power $\vpower_\leak$ can be well approximated with a normal distribution according to the central limit theorem \cite{durrett2010}. We assume that the nominal leakage power $\vmean{\leak}$ and covariance matrix $\covariance{\vpower_\leak}$ are given. Consequently, an intermediate version of the leakage model is the following:
\[
  \vpower_\leak = \vmean{\leak} + \vnormal_\leak
\]
where $\vnormal_\leak \sim \normal(0, \covariance{\vpower_\leak})$.

Due to the fact that the leakage power has a strong dependency on temperature, the later should be considered in the leakage modeling. As it was shown in \cite{liu2007}, a linear approximation of the leakage current yields sufficiently accurate results. Therefore, we also employ this technique in the paper and assume that $\vmean{\leak}$ is given at the reference temperature $\vtemperature_\refer$ and the coefficient of proportionality, denoted by $\vratio_\leak$, which relates the deviation from the reference temperature and the change in the leakage power, is available\footnote{$\vratio_\leak$ can be obtained through a curve fitting using, for instance, the least squares regression \cite{press2007}.}, resulting in the following final expression for the leakage power:
\begin{align} \equlabel{leakage-power}
  \vpower_\leak(\time) & = \vmean{\leak} + \vnormal_\leak + \diag{\vratio_\leak}(\vtemperature(\time) - \vtemperature_\refer) \\
  & = \vmean{\leak} + \vnormal_\leak + \ddeviation{\leak}(\vtemperature(\time) - \vtemperature_\refer) \nonumber
\end{align}
Note that both the current temperature $\vtemperature(\time)$ and reference one $\vtemperature_\refer$ are relative to the ambient temperature $\vtemperature_\amb$ as it was discussed in \secref{thermal-model}.

\subsection{Resulting Temperature}
Summing up \equref{dynamic-power} and \equref{leakage-power}, the total power dissipation can be computed as the following:
\begin{align} \equlabel{total-power}
  \vpower(\time) & = \vpower_{\dyn}(\time) + \vpower_{\leak}(\time) \\
  & = \vmean{\dyn}(\time) + \vmean{\leak} - \ddeviation{\leak} \vtemperature_\refer + \ddeviation{\leak} \vtemperature(\time) + \nonumber \\
  & { } \qquad + \ddeviation{\dyn}(\time) \vnormal_\dyn + \vnormal_\leak \nonumber
\end{align}
Therefore, we rewrite \equref{fourier} as follows:
\begin{align} \equlabel{fourier-pv}
  \mcapacitance \frac{d\vtemperature(\time)}{d\time} + \amconductance \vtemperature(\time) = \vmean{}(\time) + \ddeviation{\dyn}(\time) \vnormal_\dyn + \vnormal_\leak
\end{align}
where:
\begin{align*}
  & \a{\mconductance} = \mconductance - \ddeviation{\leak} \\
  & \vmean{}(\time) = \vmean{\dyn}(\time) + \vmean{\leak} - \ddeviation{\leak} \vtemperature_\refer
\end{align*}


  \section{Model of the Environmental Noise} \seclabel{noise}
  Suppose the system at each moment of time is exposed to a random noise, denoted by $\noise(\time)$, which affects the total power dissipation and, consequently, the resulting temperature. In this paper, we assume that $\noise(\time)$ is the white noise with a known covariance matrix $\cov{\noise}$.

Due to the assumptions, $\noise(\time)$ is a \mnrv\ at each moment of time $\time$, i.e., $\noise(\time) \sim \normal(\vzero, \cov{\noise})$. We decompose the noise as $\noise(\time) = \ddev{\noisy} \vnormal_\noisy(\time)$, where $\ddev{\noisy} = \factorize{\cov{\noise}}$ and the stochastic process $\{ \vnormal_\noisy(\time) \}_{\time \in \timeset}$ is the standard $\nodecount$-dimensional white noise \cite{oksendal2003}, i.e., a set of vectors of independent \snrvs\ $\vnormal_\noisy(\time) \sim \normal(\vzero, \mone)$. Hence, using \equref{fourier-constant-power}, we have the following stochastic process within one time interval of constant power:
\begin{equation} \equlabel{fourier-noise}
  \mcapacitance \frac{d\vtemp(\time)}{d\time} + \mconductance \vtemp(\time) = \vpower + \ddev{\noisy} \vnormal_\noisy(\time)
\end{equation}
\equref{fourier-noise} can be rewritten in the differential form:
\begin{equation} \equlabel{fourier-wiener}
  d\vtemp(\time) = \mcapacitance^{-1} (\vpower - \mconductance \vtemp(\time))d\time + \mcapacitance^{-1} \ddev{\noisy} d\m{W}(\time)
\end{equation}
where $d\m{W}(\time) = \vnormal_\noisy(\time) d\time$ with $\{ \m{W}(\time) \}_{\time \in \timeset}$ being the Wiener process or Brownian motion \cite{oksendal2003}. The integration of \equref{fourier-wiener} cannot be done in the regular Riemann-Stieltjes sense, since the Brownian motion is nowhere differentiable, therefore, a special calculus should be applied. We follow the It\^{o} interpretation \cite{oksendal2003} of the stochastic differential equation in \equref{fourier-wiener}. It can be seen that the stochastic process $\{ \vtemp(\time) \}_{\time \in \timeset}$ is a multidimensional Ornstein-Uhlenbeck process \cite{kloeden1992} driven by the Brownian motion. In order to solve \equref{fourier-wiener}, we apply the It\^{o} formula \cite{oksendal2003} to the function $\ecg{}{t} \vtemp(\time)$ and obtain:
\begin{align*}
  d(\ecgt \vtemp(\time)) & = \cg \ecgt \vtemp(\time) d\time + \ecgt d\vtemp(\time) \\
  & = \ecgt \mcapacitance^{-1} (\vpower d\time + \ddev{\noisy} d\v{W}(\time))
\end{align*}
The solution can be found by taking integrals on both sides:
\begin{equation} \equlabel{solution-full}
  \vtemp(\time) = \m{A}(\time) \vtemp(0) + \m{B}(\time) \vpower + \v{D}(\time)
\end{equation}
where, to simplify the further derivation, we introduce the following notation:
\begin{equation} \equlabel{d}
  \v{D}(\time) = \int_0^{\time} \ecg{}{(s - \time)} \mcapacitance^{-1} \ddev{\noisy} \: d\m{W}(s)
\end{equation}
Since the Wiener process has independent normally distributed increments, an integration with respect to it, i.e., $\int_0^\time f(s) dW(s)$, is a \nrv. It can also be shown that in this case the mean is zero and variance is equal to $\int_0^\time f^2(s) ds$. Hence, $\v{D}(\time) \sim \normal(\vzero, \cov{\v{D}(\time)})$, i.e., $\v{D}(\time)$ is a \mnrv\ with a zero mean vector and the following covariance matrix:
\begin{align}
  \cov{\v{D}(\time)} & = \int_0^\time \ecg{2}{(s - \time)} (\mcapacitance^{-1} \ddev{\noisy})^2 \; ds \nonumber \\
  & = (2 \cg)^{-1} (\mone - \emcgt{2}) (\mcapacitance^{-1} \ddev{\noisy})^2 \equlabel{covariance-d}
\end{align}


  % \section{Static Steady-State Temperature Analysis} \seclabel{sssta}
  % Suppose that we are given a periodic dynamic power profile and are interested in the thermal behaviour of the system in the long run. In this case, it is assumed that, after a sufficiently long period of time, the system reaches such a state, where temperature either preserves a constant value or exhibits a periodic pattern. Concequently, the state is called either the \definition{static steady state} or \definition{dynamic steady state}, respectiely. Therefore, there are two types of the steady-state temperature analysis: \definition{static steady-state temperature analysis} (SSSTA), discussed in this section, and \definition{dynamic steady-state temperature analysis} (DSSTA), discussed in \secref{dssta}.

The SSSTA delivers a single vector of temperature values for each of the thermal nodes of the system.

\subsection{Problem Formulation}
The source of uncertainties is the process variation.

Given:
\begin{itemize}
  \item A multiprocessor system $\system$.
  \item A nominal dynamic power profile $\powerprofile_\dyn$.
  \item The variance $\deviation{ij}^2$ of $\power_{ij} \; \forall i \in \nodeindex, j \in \timeindex$.
\end{itemize}

Find:
\begin{itemize}
  \item The probability distribution of the temperature profile $\temperatureprofile$ with respect to $\powerprofile$, when the static steady state is reached.
\end{itemize}

\subsection{Solution} \seclabel{sss-solution}
Additional assumptions:
\begin{itemize}
  \item $\power_{ij} \sim \normal(\mean{ij}, \deviation{ij}^2)$, i.e., the power dissipation is distributed normally.
  \item $\power_{i_1 j_1}$ and $\power_{i_2 j_2}$ are independent $\forall i_1, i_2 \in \nodeindex,  i_1 \neq i_2$ and $\forall j_1, j_2 \in \timeindex$.
\end{itemize}

As discussed earlier, the system is supposed to reach a hypothetical thermal balance, where temperature does not change over the time. In this case, the power profile is averaged and given as a constant input to the thermal RC circuit. Therefore, \equref{fourier} does not depend on time and drops the derivative term:
\begin{equation} \equlabel{sss-fourier}
  \a{\vtemperature} = \mconductance^{-1} \a{\vpower}
\end{equation}
where $\a{\vpower}$ is a random vector that represents the average power dissipation of thermal nodes, consequently, $\a{\vtemperature}$ is a random vector that corresponds to the average temperature. As mentioned in \secref{thermal-model}, the ambient temperature, $\vtemperature_\amb$, should be added to the right-hand side of \equref{sss-fourier} in order to obtain the actual temperature values.

$\a{\vpower}$ is computed with an ordinal equation:
\[
  \a{\vpower} = \frac{1}{\stepcount} \sum_{j \in \timeindex} \vpower_j
\]
where $\vpower_j$ are \mnrvs. As mentioned in \secref{uncertainties}, \rvs\ $\power_{ij}$ for a fixed $i$ and $\forall j$ (a row in the power matrix $\mpower$) are required to jointly deviate from the nominal (mean) values into the same direction. Therefore:
\[
  \a{\vpower} = \frac{1}{\stepcount} \sum_{j \in \timeindex} \vmean{j} + \frac{1}{\stepcount} \sum_{j \in \timeindex} \ddeviation{j} \; \vstdnormal
\]
It can be seen that $\a{\vpower}$ is a \mnrv\ with independent components:
\begin{align}
  & \a{\vpower} \sim \normal(\expectation{\a{\vpower}}, \covariance{\a{\vpower}}) \nonumber \\
  & \expectation{\a{\vpower}} = \frac{1}{\stepcount} \sum_{j \in \timeindex} \vmean{j} \equlabel{steady-power-mean} \\
  & \covariance{\a{\vpower}} = \frac{1}{\stepcount^2} \left[ \sum_{j \in \timeindex} \ddeviation{j} \right]^2 \equlabel{steady-power-covariance}
\end{align}
Consequently, after the linear transformation of $\a{\mpower}$ in \equref{sss-fourier}, $\a{\mtemperature}$ also becomes a \mnrv:
\begin{align*}
  & \a{\vtemperature} \sim \normal(\expectation{\a{\vtemperature}}, \covariance{\a{\vtemperature}}) \\
  & \expectation{\a{\vtemperature}} = \mconductance^{-1} \expectation{\a{\vpower}} \\
  & \covariance{\a{\vtemperature}} = \mconductance^{-1} \covariance{\a{\vpower}} (\mconductance^{-1})^T
\end{align*}
where $\expectation{\a{\vpower}}$ and $\covariance{\a{\vpower}}$ are found using \equref{steady-power-mean} and \equref{steady-power-covariance}, respectively.


  \section{Transient Temperature Analysis} \seclabel{tta}
  \subsection{Problem Formulation}
Consider a multiprocessor system affected by a random noise. The nominal parameters of the platform, application, and ambience are assumed to be known. However, due to the noise (caused by, for instance, the process variation, various ambience conditions, etc.) the actual dynamic power dissipation and, consequently, temperature fluctuations are uncertain. The noise is represented by a multivariate normal random variable, symbolically denoted as $\noise$, with $\nodecount$ components corresponding to each of the thermal node, i.e., $\noise \sim \normal(\vzero, \covariance{\noise})$.

Given:
\begin{itemize}
  \item A multiprocessor platform $\platform$.
  \item Nominal physical parameters $\physics$.
  \item A nominal dynamic power profile $\powerprofile$.
  \item The covariance matrix of the noise $\covariance{\noise}$.
\end{itemize}

Find:
\begin{itemize}
  \item The probability distribution of the (stochastic) temperature profile $\temperatureprofile$ of the platform with respect to $\powerprofile$.
\end{itemize}

\subsection{Model}
Since the input to the analysis is a discrete power profile, the nominal power is assumed to be fixed within each of the time intervals $\timepartition$. Therefore, when modeling one time interval, $\vpower(t)$ is constant in \equref{fourier}, i.e., $\vpower(t) = \vpower$:
\[
  \mcapacitance \frac{d\vtemperature(\time)}{d\time} + \mconductance \vtemperature(\time) = \vpower
\]
In order to model the presence of noise, we introduce an additional term, $\noise$, in the right-hand side of the equation above. $\noise$ is decomposed into $\mnoisedeviation \vnoise_t$, where $\vnoise_t$ is a vector of independent standard normal random variables, i.e., $\vnoise_\time \sim \normal(\vzero, \mone)$, and  $\mnoisedeviation = \m{U} \m{\Lambda}^{1/2}$. The matrices $\m{U}$ and $\m{\Lambda}$ are found using the eigenvalue decomposition of the covariance matrix, i.e., $\covariance{\noise} = \m{U} \m{\Lambda} \m{U}^T$. Note that $\mnoisedeviation \mone \mnoisedeviation^T = \covariance{\noise}$, therefore, $\m{U} \m{\Lambda}^{1/2} \vnoise_t \sim \normal(\vzero, \covariance{\noise})$. Hence, denoting time-dependent (stochastic) processes with the subscript $\time$, we have:
\begin{equation} \equlabel{fourier-noise}
  \mcapacitance \frac{d\vtemperature_\time}{d\time} + \mconductance \vtemperature_\time = \vpower + \mnoisedeviation \vnoise_t
\end{equation}
The stochastic process $\{ \vnoise_\time \}_{\time \in \timeset}$ is known as the white noise \cite{oksendal2003}, in this case, multidimensional. \equref{fourier-noise} can be rewritten in the differential form:
\begin{equation} \equlabel{fourier-wiener}
  d\vtemperature_\time = \mcapacitance^{-1}(\vpower - \mconductance \vtemperature_\time)d\time + \mcapacitance^{-1} \mnoisedeviation \vnoise_t d\time
\end{equation}
Note that $\vnoise_\time d\time = d\m{W}_\time$ where $\{ \m{W}_\time \}_{\time \in \timeset}$ is the Wiener process \cite{oksendal2003}. Also it can be seen that the stochastic process $\{ \vtemperature_\time \}$ resembles the Ornstein-Uhlenbeck process \cite{kloeden1992}. The It\^{o} interpretation \cite{oksendal2003} of the last equations is a stochastic process $\{ \vtemperature_\time \}_{\time \in \timeset}$ that satisfies the following integral equation:
\[
  \vtemperature_\time = \vtemperature_0 + \int_0^\time a(s, \vtemperature_s)ds + \int_0^\time b(s, \vtemperature_s) d\m{W}_\time
\]
In order to solve \equref{fourier-wiener} and find $\vtemperature_\time$, we apply the It\^{o} formula to the function $\ecg{}{t} \vtemperature_\time$, where $e^\m{A}$ is the matrix exponential of $\m{A}$, and obtain:
\begin{align*}
  d(\ecgt \mtemperature_\time) & = \cg \ecgt \vtemperature_\time d\time + \ecgt d\vtemperature_\time \\
  & = \ecgt \mcapacitance^{-1} (\vpower d\time + \mnoisedeviation d\v{W}_t)
\end{align*}
The solution can be found by taking integrals on both sides:
\begin{align} \equlabel{solution-full}
  \vtemperature_\time = & \emcgt \vtemperature_0 - \\
    & (\cg)^{-1}(\emcgt - \mone) \mcapacitance^{-1} \vpower + \nonumber \\
    & \int_0^\time \ecg{}{(s - \time)} \mcapacitance^{-1} \mnoisedeviation \; d\m{W}_s \nonumber
\end{align}
To simplify the further derivation, we introduce the following notation:
\begin{align}
  & \m{A}(\time) = \emcg{\time} \equlabel{a} \\
  & \m{B}(\time) = -(\cg)^{-1}(\emcg{\time} - \mone) \mcapacitance^{-1} \equlabel{b} \\
  & \v{D}_\time = \int_0^{\time} \ecg{}{(s - \time)} \mcapacitance^{-1} \mnoisedeviation \: d\m{W}_s \equlabel{d}
\end{align}
Therefore, \equref{solution-full} can be rewritten:
\[
  \vtemperature_\time = \m{A}(\time) \vtemperature_0 + \m{B}(\time) \vpower + \v{D}_\time
\]
Since the Wiener process has independent normally distributed increments, an integration with respect to it, i.e., $\int_0^\time f(s) dW_s$, is a normal random variable. It can also be shown that in this case the mean is zero and variance is equal to $\int_0^\time f^2(s) ds$. Hence, $\v{D}_\time \sim \normal(\vzero, \covariance{\v{D}_\time})$ is a multivariate normal random variable with a zero mean vector and the following covariance matrix:
\begin{align}
  \covariance{\v{D}_\time} & = \int_0^\time \ecg{}{(s - \time)} \mcapacitance^{-1} \mnoisedeviation \; ds \nonumber \\
    & = (\cg)^{-1} (\mone - \emcgt) \mcapacitance^{-1} \mnoisedeviation \equlabel{covariance-d}
\end{align}
The components of $\v{D}_\time$ in general are not independent. Consequently, $\m{B}(\time) \vpower + \v{D}_\time \sim \normal(\m{B}(\time) \vpower, \: \covariance{\v{D}_\time})$.

The solution given by \equref{solution-full} is applicable for one time interval with constant $\vpower$. In order to model the whole period $\totaltimeinterval$, the computations should be performed for each of the subintervals sequentially. Therefore, we have the following recurrent expression:
\begin{equation} \equlabel{recurrence}
  \vtemperature_{\time_{i + 1}} = \m{A}(\timeinterval_i) \vtemperature_{\time_i} + \m{B}(\timeinterval_i) \vpower_{\time_i} + \m{D}_{\timeinterval_i}
\end{equation}
Here $\vtemperature_{\time_i}$ is the multivariate normal random variable obtained in the previous step of the iterative process. Thus, $\m{A}(\timeinterval_i) \vtemperature_{\time_i}$ is a multivariate normal variable as well with:
\begin{align*}
  & \mean{\m{A}(\timeinterval_i) \vtemperature_{\time_i}} = \m{A}(\timeinterval_i)\mean{\vtemperature_{\time_i}} \\
  & \covariance{\m{A}(\timeinterval_i) \vtemperature_{\time_i}} = \m{A}(\timeinterval_i) \covariance{\vtemperature_{\time_i}} \m{A}^T(\timeinterval_i)
\end{align*}
Due to the properties of the Wiener process, $\vtemperature_{\time_i}$ is independent of $\m{D}_{\timeinterval_i}$. Therefore, $\vtemperature_{\time_{i+1}}$ is a multivariate normal random variable with the following mean and covariance:
\begin{align*}
  & \mean{\vtemperature_{\time_{i+1}}} = \m{A}(\timeinterval_i)\mean{\vtemperature_{\time_i}} + \m{B}(\timeinterval_i) \vpower_{\time_i} \\
  & \covariance{\vtemperature_{\time_{i+1}}} = \m{A}(\timeinterval_i) \covariance{\vtemperature_{\time_i}} \m{A}^T(\timeinterval_i) + \covariance{\m{D}_{\timeinterval_i}}
\end{align*}
\equref{a}, \equref{b}, and \equref{covariance-d} can be used to perform the computations.


  \section{Dynamic Steady-State Temperature Analysis} \seclabel{dssta}
  As in the SSSTA (\secref{sssta}), the given power profile is assumed to be periodic. However, instead of aiming at a hypotherical thermal balance with a single constant temperature for each of the thermal nodes, the \definition{dynamic steady-state temperature analysis} (DSSTA) delivers a set of periodic curves, similar to the TTA (\secref{tta}), that describe a periodic pattern of the thermal behaviour of the system.

\subsection{Problem Formulation}
The sources of uncertainties are a random noise and the process variation.

Given:
\begin{itemize}
  \item A multiprocessor system $\system = (\platform, \physics)$.
  \item A nominal dynamic power profile $\powerprofile$.
  \item Knowledge about uncertainties presented in the system (specified in the following subsections).
\end{itemize}

Find:
\begin{itemize}
  \item The probability distribution of the periodic temperature profile $\temperatureprofile$ with respect to $\powerprofile$.
\end{itemize}

\subsection{Solution with Noise}
Additional assumptions and givens are similar to those in \secref{ta-noise}.

\subsection{Solution with Process Variation}
Additional assumptions and givens are similar to those in \secref{ta-process-variation}. Since the resulting temperature curves should be periodic, the following boundary condition is to be satisfied:
\begin{equation} \equlabel{dss-boundary}
  \vtemperature_0 = \vtemperature_{\stepcount}
\end{equation}
Using \equref{ta-fourier-non-recurrent}, the temperature vector at the end of the curves can be computed as:
\begin{align*}
  \vtemperature_{\stepcount} & = \am{A}_{\stepcount - 1} \vtemperature_0 + \m{F}_{\stepcount - 1} + \m{H}_{\stepcount - 1} \vstdnormal
\end{align*}
Taking into consideration the boundary condition given by \equref{dss-boundary}, the solution of the last equation with respect to $\vtemperature_0$ is the following:
\[
\vtemperature_0 = (\mone - \am{A}_{\stepcount - 1})^{-1} (\m{F}_{\stepcount - 1} + \m{H}_{\stepcount - 1} \vstdnormal)
\]
Consequently, the mean and covariance at the beginning of the periodic power profile are:
\begin{align*}
  & \vtemperature_0 \sim \normal(\expectation{\vtemperature_0}, \covariance{\vtemperature_0}) \\
  & \expectation{\vtemperature_0} = (\mone - \am{A}_{\stepcount - 1})^{-1} \v{F}_{\stepcount - 1} \\
  & \covariance{\vtemperature_0} = (\mone - \am{A}_{\stepcount - 1})^{-1} \v{H}_{\stepcount - 1} \v{H}_{\stepcount - 1}^T (\mone - \am{A}_{\stepcount - 1}^T)^{-1}
\end{align*}
The rest of the temperature vectors is successively found using \equref{ta-fourier-recurrence}, hence:
\begin{align*}
  & \vtemperature_{i + 1} \sim \normal(\expectation{\vtemperature_{i + 1}}, \covariance{\vtemperature_{i + 1}}), \sep i \in \timeindex \setminus \{\stepcount - 1\} \\
  & \expectation{\vtemperature_{i + 1}} = \m{A}_i \expectation{\vtemperature_i} + \m{B}_i \vmean{i} \\
  & \covariance{\vtemperature_{i + 1}} = \m{A}_i \covariance{\vtemperature_i} \m{A}_i^T + \m{B}_i \ddeviation{i}^2 \m{B}_i^T \\
  & {} \qquad \qquad + \m{A}_i \covariance{\vtemperature_i, \vpower_i} \m{B}_i^T + \m{B}_i \covariance{\vtemperature_i, \vpower_i}^T \m{A}_i^T
\end{align*}


  \bibliographystyle{unsrt}
  \bibliography{include/references}
\end{document}
